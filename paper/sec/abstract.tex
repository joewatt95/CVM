\begin{abstract}
Estimating the number of distinct elements in a data stream is a classical problem with numerous applications in computer science.
We formalize a recent, remarkably simple, randomized algorithm for this problem due to Chakraborty, Vinodchandran, and Meel (the so-called CVM algorithm).
Their algorithm deviated considerably from the state-of-the art, due to its avoidance of intricate derandomization techniques, while still maintaining a close to optimal logarithmic space complexity.

Central to our formalization is a new proof technique based on \emph{functional probabilistic invariants}, which allows us to derive concentration bounds using the Cram\'{e}r--Chernoff method.
This simplifies the formal analysis considerably compared to the original transformation-based proof.
Moreover, our technique opens up the possible algorithm design space; we demonstrate this by introducing and verifying with a new variant of the CVM algorithm that is total and unbiased---neither property is possessed by the original algorithm.
In this paper, we introduce the proof technique, describe its use in Isabelle/HOL to mechanize both versions of the CVM algorithm, and present a supporting formalized library on negatively associated random variables used to verify the latter variant.
\end{abstract}


