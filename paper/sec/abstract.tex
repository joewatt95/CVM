\begin{abstract}
In 2022, Chakraborty et al.~\cite{chakraborty2022} published a streaming
algorithm (henceforth, the CVM algorithm) for the distinct
elements problem, that deviated considerably from the state-of-the art, due to its simplicity
and avoidance of standard derandomization techniques, while still maintaining a close to optimal
logarithmic space complexity.

In this entry, we verify the CVM algorithm's correctness using a new technique which simplifies
the formal analysis considerably compared to the orignal proof by Chakraborty et
al. The main idea is based on a probabilistic invariant that allows us to derive concentration bounds
using the Cram\'{e}r--Chernoff method.

This new technique opens up the possible algorithm design space, and we introduce a new variant of the
CVM algorithm, that is total, and also has an additional property in addition
to concentration: unbiasedness. This means the expected result of the algorithm is exactly equal to
the desired result. The latter is also a new property, that neither the original CVM algorithm
nor classic algorithms for the distinct elements problem possess.
\end{abstract}


