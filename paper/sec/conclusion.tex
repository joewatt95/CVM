\section{Conclusion}\label{sec:conclusion}
We presented the first formalization of the CVM algorithm using Isabelle/HOL.
Central to our formalization is a novel invariant-based proof technique to establish exponentially decreasing tail-bounds for randomized algorithms, which is inspired by our alternative analysis of the CVM algorithm via the Cram\'{e}r--Chernoff method.
Our technique can be summarized by the following two steps:
\begin{enumerate}
\item Find functionals over the state distribution of the algorithm, for which it is possible to establish bounds on their expectation inductively/recursively.
\item Use those bounds to establish tail bounds on the result of the algorithm.
\end{enumerate}
Comparing our approach against the original proof by Chakraborty et al.~\cite{chakraborty2023} shows that our technique yields a considerably shorter formalization (with \locnew~vs.~\locold~lines).
Interestingly, our technique also readily generalized to a new CVM variant with stronger properties (totality and unbiasedness)---we formalized this latter version using the same invariant, together with a new library of results for negative association.
In future work, it would be interesting to formalize other variations of subsampling for CVM.

Note that we have yet to apply our proof technique to examples beyond CVM. (It is easy to construct artificial examples.)
Identifying realistic applications for our new method is an interesting avenue for future work---this could lead to further refinements of the method, and a better understanding of how to identify suitable functionals for the proofs.
