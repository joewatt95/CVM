\section{Conclusion}\label{sec:conclusion}
We presented the first formalization of the CVM algorithm using Isabelle/HOL.
Central to our formalization is a novel invariant-based proof technique to establish exponentially decreasing tail-bounds for randomized algorithms, which is inspired by our alternative analysis of the CVM algorithm via the Cram\'{e}r--Chernoff method; comparing our approach against the original proof by Chakraborty et al.~\cite{chakraborty2023} shows that our technique yields a considerably shorter formalization (with \locnew~vs.~\locold~lines).
Interestingly, our technique also readily generalized to a new CVM variant with stronger properties (totality and unbiasedness)---we formalized this latter version using the same invariant, together with a new library of results for negative association.
In future work, it would be interesting to formalize other variations of subsampling for CVM.

Our novel technique can be summarized by the following two steps:
\begin{enumerate}
\item Find functions of the state of the algorithm, for which it is possible to establish upper bounds of their expectation recursively. 
\item Use those bounds to establish tail bounds on the result of the algorithm.
\end{enumerate}
However, we would like to note that, we do not yet have good further examples, where our approach can be applied to. (It is easy to construct artificial examples.)
This also means that our definition of the method may require refinement.
Another issue is that we do not have general strategies to identify the functionals to consider.
Identification of realistic applications for this new method is an interesting avenue for future work.
