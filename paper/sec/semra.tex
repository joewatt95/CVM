\section{Semantics of randomized algorithms}
It is useful to briefly review how reasoning about randomized algorithms works in Isabelle using the Giry monad.
The key idea is to model a randomized algorithm as a probability space representing the distributions of its result.
Let us, e.g., consider \cref{alg:example}.
\begin{algorithm}[h!]
\caption{Example for sequential composition.}\label{alg:example}
\begin{algorithmic}[1]
\State $p \getsr \Ber(\frac{1}{2})$
\State $q \getsr \Ber(\frac{1}{3}+p)$
\State \Return $q$
\end{algorithmic}
\end{algorithm}%

In the first step the algorithm flips a fair coin, such that $p$ is $1$ with probability $\frac{1}{2}$ and $0$ otherwise.
In the second step the algorithm flips a coin $q$ which is $1$ with probability $\frac{1}{3}$, if $p=0$ or $\frac{2}{3}$ if $q=1$.
I.e., $q$ is a compound distribution a half-half combination of $\Ber(\frac{1}{3})$ and $\Ber(\frac{2}{3})$,

Any algorithm can be represented using only two combinators, the bind and return combinator, as well as, primitive random operations.

\paragraph*{Primitive Random Operations}
For example a simple fair coin flip is represented using the Bernoulli distribution $\Ber(\frac{1}{2})$.

\paragraph*{Return Combinator}
Given a singleton element $x$, we can construct the singleton probabilty space, e.g., we assign the probability $1$ to $x$ and $0$ to everything else.
In notation, this is written as: $\mathrm{return}\, x$. 

\paragraph*{Bind Combinator}
The bind combinator represents sequential composition of two randomized algorithms $m$ and $f$, where the second randomized algorithm consumes the output of the first.
In notation, this can be expressed using $m \isa{\isasymbind} f$.
Mathematically, this is the most involved operation, because $f$ is a function returning probability spaces, which are then mixed using the first probability space $m$.

Let us consider an event $A$ in the probability space $m \isa{\isasymbind} f$, then its probability can be evaluated by integrating over its probabilities in $f$ with respect to $m$:
\[
  \prob_{m \isa{\isasymbind} f} (A) = \int_m \prob_{f(x)} (A) \, d x \textrm{.}
\]
Another interesting property is what we mentioned before in the introduction. If $h$ is a random variable over $m \isa{\isasymbind} f$, we can compute its expectation as:
\begin{equation}
  \label{eq:integral_bind}
  \expect_{m \isa{\isasymbind} f} [h] = \int_m \expect_{f(x)} [h] \, d x \textrm{.}
\end{equation}

\todo{TODO: Add external references and what we have glossed over (infinite loops, measurability.)}
