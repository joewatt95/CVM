\section{Background}
\label{sec:background}

\subsection{Randomized Algorithms and Distinct Elements}

The CVM algorithm is a \emph{streaming} algorithm for the distinct elements problem.
As shown in~\cref{alg:cvm}, given a data stream $a_1,\dots, a_l$, the goal of such algorithms is to return an accurate cardinality estimate for the set $A = \{a_1,\dots,a_l\}$.

Importantly, CVM is a \emph{probably approximately correct} (PAC) algorithm where its output estimate $R$ satisfies
$\prob \left( |R - |A| | > \varepsilon |A| \right) \leq \delta$
for parameters $\varepsilon$ and $\delta$,
i.e., the probability that the relative error of $R$ with respect to $|A|$ exceeds $\varepsilon$ is at most $\delta$. % I think here it's good to have both a text and formal description.
Moreover, let us assume that the space needed to store each element in the stream is $b$ bits, then the CVM algorithm requires only $\bigo(\varepsilon^{-2} b \ln(\delta^{-1} l))$ bits of mutable state, which is far less than storing each stream element deterministically.

\begin{remark}
The asymptotically optimal randomized algorithm for distinct elements requires $\bigo( \varepsilon^{-2} \ln \delta + b)$ bits, but it requires more advanced algorithmic techniques. It would not be possible to present using such elementary steps as in \cref{alg:cvm} as it involves computations in finite fields and random walks in expander graphs~\cite{blasiok2020, karayel2023}.
\lipicsEnd\end{remark}

\subsection{Semantics of Randomized Algorithms}
We briefly review how reasoning about randomized algorithms works in Isabelle/HOL using the Giry monad~\cite{giry1982}.
Multiple authors provide more thorough discussions of the concept in the context of Isabelle and other proof assistants~\cite{audebaud2009,eberl2020, lochbihler2016}.

The key idea is to model a randomized algorithm as a probability space representing the distribution of its results.
As an example, let us consider \cref{alg:example}.
\begin{algorithm}[h!]
\caption{Example for sequential composition.}\label{alg:example}
\begin{algorithmic}[1]
\State $p \getsr \Ber(\frac{1}{2})$
\State $q \getsr \Ber(\frac{1}{3}+\frac{p}{2})$
\State \Return $q$
\end{algorithmic}
\end{algorithm}%

In the first step,~\cref{alg:example} flips a fair coin, such that $p$ is $1$ with probability $\frac{1}{2}$ and $0$ otherwise; the notation $\Ber(p)$ represents the Bernoulli distribution.
In the second step, the algorithm flips a coin $q$ which depends on $p$.
This has the consequence that, to semantically model $q$, we have to consider functions returning probability spaces, like: $p \mapsto \Ber(\frac{1}{3}+\frac{p}{2})$, which is being \emph{bound} to the distribution of $p$.
The resulting distribution for $q$ is a \emph{compound distribution} resulting from a combination of $\Ber(\frac{1}{3})$ (when $p = 0$) and $\Ber(\frac{5}{6})$ (when $p = 1$).

This example captures the main aspects of modeling randomized algorithms in the Giry monad.
Indeed, randomized algorithms can be modeled using the following ingredients:

\begin{description}
\item[Primitive Random Operations.] For example, a simple fair coin flip is represented using the Bernoulli distribution, $\Ber(\frac{1}{2})$.
\item[Return Combinator.]
Given an element $x$, we can construct the singleton probability space, assigning probability $1$ to $x$ and $0$ to everything else.
In monad notation, this is written as: $\mathrm{\bf return}\, x$.

\item[Bind Combinator.]
The bind combinator represents sequential composition of two randomized algorithms $m$ and $f$, where the latter randomized algorithm consumes the output of the former; in monad notation, this is: $m \isa{\isasymbind} f$.
Mathematically, this is the most involved operation, because $f$ is a function returning probability spaces, which takes inputs from the probability space $m$.

Let us consider an event $A$ in the probability space $m \isa{\isasymbind} f$.
Its probability can be evaluated by integrating over its probabilities in $f$ with respect to $m$:
\[
  \prob_{m \isa{\isasymbind} f} (A) = \int_m \prob_{f(x)} (A) \, d x \textrm{.}
\]

Another key property is the calculation of expectations;
if $h$ is a random variable over $m \isa{\isasymbind} f$, we can compute its expectation as:
\begin{equation}
  \label{eq:integral_bind}
  \expect_{m \isa{\isasymbind} f} [h] = \int_m \expect_{f(x)} [h] \, d x \textrm{.}
\end{equation}

\cref{eq:integral_bind} is crucially used to establish the invariants we introduce next in~\cref{sec:invariants}.
\end{description}
