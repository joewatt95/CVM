\section{Formalization of the CVM algorithm}\label{sec:formalization}
Let us now turn to details of our formalization of the CVM algorithm in Isabelle/HOL using our invariant-based approach.
We verified both the total, unbiased variant (\cref{alg:cvm_new}) and the original variant (\cref{alg:cvm}) from the introduction.

%TODO: make sure these names are correct!
\begin{note}
In our supplementary material, the invariant-based verification approach is in the directory \verb|CVM_Invariant|.
The theory \verb|CVM_Abstract_Algorithm| verifies a generalized version of the CVM algorithm, with an abstract subsampling operation that is required to fulfill \cref{i:subsample_condition}.
The specialization happens in the following theories, where \verb|CVM_Original_Algorithm| verifies the original algorithm, and \verb|CVM_New_Unbiased_Algorithm| verifies the new total and unbiased variant.
Note that only \verb|CVM_New_Unbiased_Algorithm| depends on the new library for negatively associated random variables, which we describe in more detail in \cref{sec:formalization_neg_dep}.
The total number of lines required for the verification of the original algorithm is \locnew~lines.
In addition, we actually verified a slight generalization of~\cref{alg:cvm} where the subsampling probability can be any $f \in [\frac{1}{2};e^{-1/12}]$; the original CVM algorithm~\cite{chakraborty2022} is the special case $f=\frac{1}{2}$.
\lipicsEnd\end{note}

%Our formalization also allows $n \geq 12 \varepsilon^{-2} \ln(6 l \delta^{-1})$, i.e., $n$ can be any integer value greater than the bound.
%Another deviation is that we have better constant factors compared to the original presentation---the buffer size used by Chakraborty et al. is $12 \varepsilon^{-2} \log_2(8 l \delta^{-1}) \approx 17.3 \varepsilon^{-2} \ln(8 l \delta^{-1})$.
%This difference is mainly due to the fact that the invariant-based proof does not require conditioning on particular values of $p$ (the subsampling probability at the end of the algorithm) in contrast to the transformation-based proof which applies Chernoff bounds for each possible value of $p$ (down to some lower bound $q$) and establishes the final error probability as a union bound.
%However, we think neither Chakraborty et al., nor we, have tried to optimize constants and it should be possible to find constant factor improvements using a more careful analysis in both proofs.
% It feels weird to imply that Chakraborty et al. did not try to optimize. Added qualifier to make sure "we think" that they didn't.

\begin{figure}[t]
\begin{isabelle_cm}
\isacommand{context}\isamarkupfalse%
\isanewline
\ \ \isakeyword{fixes}\ f\ {\isacharcolon}{\kern0pt}{\isacharcolon}{\kern0pt}\ real\ \isakeyword{and}\ n\ {\isacharcolon}{\kern0pt}{\isacharcolon}{\kern0pt}\ nat\isanewline
\ \ \isakeyword{assumes}\ f{\isacharunderscore}{\kern0pt}range{\isacharcolon}{\kern0pt}\ {\isacartoucheopen}f\ {\isasymin}\ {\isacharbraceleft}{\kern0pt}{\isadigit{1}}{\isacharslash}{\kern0pt}{\isadigit{2}}{\isachardot}{\kern0pt}{\isachardot}{\kern0pt}{\isacharless}{\kern0pt}{\isadigit{1}}{\isacharbraceright}{\kern0pt}{\isacartoucheclose}\ {\isacartoucheopen}n\ {\isacharasterisk}{\kern0pt}\ f\ {\isasymin}\ {\isasymnat}{\isacartoucheclose}\ \isakeyword{and}\ n{\isacharunderscore}{\kern0pt}gt{\isacharunderscore}{\kern0pt}{\isadigit{0}}{\isacharcolon}{\kern0pt}\ {\isacartoucheopen}n\ {\isachargreater}{\kern0pt}\ {\isadigit{0}}{\isacartoucheclose}\isanewline
\isakeyword{begin}\isanewline
\isanewline
\isacommand{definition}\isamarkupfalse%
\ {\isacartoucheopen}initial{\isacharunderscore}{\kern0pt}state\ {\isacharequal}{\kern0pt}\ State\ {\isacharbraceleft}{\kern0pt}{\isacharbraceright}{\kern0pt}\ {\isadigit{1}}{\isacartoucheclose}\ %
\hfill\isamarkupcmt{Setup initial state $\chi=\emptyset$ and $p=1$.\;%
}\isanewline
\isacommand{fun}\isamarkupfalse%
\ subsample\ \isakeyword{where}\ %
\hfill\isamarkupcmt{Subsampling operation: Sample random $n f$ subset.\;%
}\isanewline
\ \ {\isacartoucheopen}subsample\ {\isasymchi}\ {\isacharequal}{\kern0pt}\ pmf{\isacharunderscore}{\kern0pt}of{\isacharunderscore}{\kern0pt}set\ {\isacharbraceleft}{\kern0pt}S{\isachardot}{\kern0pt}\ S\ {\isasymsubseteq}\ {\isasymchi}\ {\isasymand}\ card\ S\ {\isacharequal}{\kern0pt}\ n\ {\isacharasterisk}{\kern0pt}\ f{\isacharbraceright}{\kern0pt}{\isacartoucheclose}\isanewline
\isanewline
\isacommand{fun}\isamarkupfalse%
\ step\ \isakeyword{where}\ %
\hfill\isamarkupcmt{Loop body.\;%
}\isanewline
\ \ {\isacartoucheopen}step\ a\ {\isacharparenleft}{\kern0pt}State\ {\isasymchi}\ p{\isacharparenright}{\kern0pt}\ {\isacharequal}{\kern0pt}\ do\ {\isacharbraceleft}{\kern0pt}\isanewline
\ \ \ \ b\ {\isasymleftarrow}\ bernoulli{\isacharunderscore}{\kern0pt}pmf\ p{\isacharsemicolon}{\kern0pt}\isanewline
\ \ \ \ let\ {\isasymchi}\ {\isacharequal}{\kern0pt}\ {\isacharparenleft}{\kern0pt}if\ b\ then\ {\isasymchi}\ {\isasymunion}\ {\isacharbraceleft}{\kern0pt}a{\isacharbraceright}{\kern0pt}\ else\ {\isasymchi}\ {\isacharminus}{\kern0pt}\ {\isacharbraceleft}{\kern0pt}a{\isacharbraceright}{\kern0pt}{\isacharparenright}{\kern0pt}{\isacharsemicolon}{\kern0pt}\isanewline
\isanewline
\ \ \ \ if\ card\ {\isasymchi}\ {\isacharequal}{\kern0pt}\ n\ then\ do\ {\isacharbraceleft}{\kern0pt}\isanewline
\ \ \ \ \ \ {\isasymchi}\ {\isasymleftarrow}\ subsample\ {\isasymchi}{\isacharsemicolon}{\kern0pt}\isanewline
\ \ \ \ \ \ return{\isacharunderscore}{\kern0pt}pmf\ {\isacharparenleft}{\kern0pt}State\ {\isasymchi}\ {\isacharparenleft}{\kern0pt}p\ {\isacharasterisk}{\kern0pt}\ f{\isacharparenright}{\kern0pt}{\isacharparenright}{\kern0pt}\isanewline
\ \ \ \ {\isacharbraceright}{\kern0pt}\ else\ do\ {\isacharbraceleft}{\kern0pt}\isanewline
\ \ \ \ \ \ return{\isacharunderscore}{\kern0pt}pmf\ {\isacharparenleft}{\kern0pt}State\ {\isasymchi}\ p{\isacharparenright}{\kern0pt}\isanewline
\ \ \ \ {\isacharbraceright}{\kern0pt}\isanewline
\ \ \ {\isacharbraceright}{\kern0pt}{\isacartoucheclose}\isanewline
\isanewline
\isacommand{fun}\isamarkupfalse%
\ run{\isacharunderscore}{\kern0pt}steps\ \isakeyword{where}\ %
\hfill\isamarkupcmt{Iterate loop over stream \isa{xs}.\;%
}\isanewline
\ \ {\isacartoucheopen}run{\isacharunderscore}{\kern0pt}steps\ xs\ {\isacharequal}{\kern0pt}\ foldM{\isacharunderscore}{\kern0pt}pmf\ step\ xs\ initial{\isacharunderscore}{\kern0pt}state{\isacartoucheclose}\isanewline
\isacommand{fun}\isamarkupfalse%
\ estimate\ \isakeyword{where}\
{\isacartoucheopen}estimate\ {\isacharparenleft}{\kern0pt}State\ {\isasymchi}\ p{\isacharparenright}{\kern0pt}\ {\isacharequal}{\kern0pt}\ card\ {\isasymchi}\ {\isacharslash}{\kern0pt}\ p{\isacartoucheclose}\isanewline
\isacommand{fun}\isamarkupfalse%
\ run{\isacharunderscore}{\kern0pt}algo\ \isakeyword{where}\ %
\hfill\isamarkupcmt{Run algorithm and estimate.\;%
}\isanewline
\ \ {\isacartoucheopen}run{\isacharunderscore}{\kern0pt}algo\ xs\ {\isacharequal}{\kern0pt}\ map{\isacharunderscore}{\kern0pt}pmf\ estimate\ {\isacharparenleft}{\kern0pt}run{\isacharunderscore}{\kern0pt}steps\ xs{\isacharparenright}{\kern0pt}{\isacartoucheclose}\isanewline
{\normalfont [\dots]}\isanewline
\isacommand{end}
\end{isabelle_cm}
\caption{Formalized version of \cref{alg:cvm_new}.}\label{alg:cvm_formalized}
\end{figure}

%The snippet also includes the relevant correctness statements, both unbiasedness and tail bounds.
A snippet of the formalization of \cref{alg:cvm_new} is presented in \cref{alg:cvm_formalized} (the formalization of \cref{alg:cvm} is very similar).
We use the same variables as in the informal presentation: $n$ for the maximal size of the buffer, $f$ for the fraction of elements to keep in the buffer when subsampling.
The condition \isa{{\isacartoucheopen}n\ {\isacharasterisk}{\kern0pt}\ f\ {\isasymin}\ {\isasymnat}{\isacartoucheclose}} expresses the requirement that the $nf$ must be integer.
Instead of representing the state using pairs, as we did in the informal discussion, we use a datatype with the single constructor \isa{State}, which has two arguments $\chi$ and $p$, the buffer and the probability that the stream elements are in the buffer, respectively.
Isabelle/HOL provides notation closely related to informal pseudocode, so it is usually feasible to read a formal statement without expert knowledge.
Nevertheless, \cref{tab:isabelle_syntax} contains a brief glossary of the syntax used in the formalization.
%\medskip
\begin{table}
\caption{Isabelle/HOL syntax used in \cref{alg:cvm_formalized}.}\label{tab:isabelle_syntax}
\noindent\begin{tabular}{l p{9cm}}
\toprule
Term & Description \\
\midrule
\isa{card\ S} & Cardinality of a finite set $S$. \\
%\isa{set\ xs} & For a sequence \isa{xs} the set of elements in the sequence. \\
%\isa{length\ xs} & For a sequence \isa{xs} the length of the sequence. \\
\isa{real} & Type of real numbers and conversion from natural numbers into real numbers. \\
\isa{nat} & Type of natural numbers (non-negative integers). \\
\isa{bernoulli{\isacharunderscore}pmf\ p} & The probability space over the Boolean values, where the probability of \isa{True} is $p$. (Bernoulli distribution.) \\
\isa{pmf{\isacharunderscore}of{\isacharunderscore}set\ S} & For a finite set $S$, the uniform probability space on $S$. (Every element of $S$ is equiprobable.) \\
\isa{map{\isacharunderscore}pmf\ f\ A} & The probability space representing the distribution of the random variable $f$ over the probability space $A$. \\
\isa{return{\isacharunderscore}pmf\ x} & The probability space of the singleton $\{x\}$. \\
\isa{foldM{\isacharunderscore}pmf\ f\ xs\ a} & Iterate randomized algorithm $f$ over the sequence $xs$ using the initial state $a$. \\
%\isa{\isasymP{\isacharparenleft}x\ in\ M{\isachardot}\ P\ x\isacharparenright} & Probability of predicate $P$ in the probability space $M$. \\
%\isa{measure{\isacharunderscore}pmf{\isachardot}expectation\ M\ f} & Expectation of the random variable $f$ over the probability space $M$. \\
\bottomrule
\end{tabular}
\end{table}
%\medskip

The theorem that establishes the correctness of the algorithm, i.e., that the relative error will exceed $\varepsilon$ with probability at most $\delta$ is expressed in the following snippet:
\begin{isabelle_cm}
\isacommand{theorem}\isamarkupfalse%
\ correctness{\isacharcolon}{\kern0pt}\isanewline
\ \ \isakeyword{assumes}\ {\isacartoucheopen}{\isasymepsilon}\ {\isasymin}\ {\isacharbraceleft}{\kern0pt}{\isadigit{0}}{\isacharless}{\kern0pt}{\isachardot}{\kern0pt}{\isachardot}{\kern0pt}{\isacharless}{\kern0pt}{\isadigit{1}}{\isacharcolon}{\kern0pt}{\isacharcolon}{\kern0pt}real{\isacharbraceright}{\kern0pt}{\isacartoucheclose}\ {\isacartoucheopen}{\isasymdelta}\ {\isasymin}\ {\isacharbraceleft}{\kern0pt}{\isadigit{0}}{\isacharless}{\kern0pt}{\isachardot}{\kern0pt}{\isachardot}{\kern0pt}{\isacharless}{\kern0pt}{\isadigit{1}}{\isacharcolon}{\kern0pt}{\isacharcolon}{\kern0pt}real{\isacharbraceright}{\kern0pt}{\isacartoucheclose}\isanewline
\ \ \isakeyword{assumes}\ {\isacartoucheopen}real\ n\ {\isasymge}\ {\isadigit{1}}{\isadigit{2}}\ {\isacharslash}{\kern0pt}\ {\isasymepsilon}\isactrlsup {\isadigit{2}}\ {\isacharasterisk}{\kern0pt}\ ln\ {\isacharparenleft}{\kern0pt}{\isadigit{3}}\ {\isacharasterisk}{\kern0pt}\ real\ {\isacharparenleft}{\kern0pt}length\ xs{\isacharparenright}{\kern0pt}\ {\isacharslash}{\kern0pt}\ {\isasymdelta}{\isacharparenright}{\kern0pt}{\isacartoucheclose}\isanewline
\ \ \isakeyword{defines}\ {\isacartoucheopen}A\ {\isasymequiv}\ real\ {\isacharparenleft}{\kern0pt}card\ {\isacharparenleft}{\kern0pt}set\ xs{\isacharparenright}{\kern0pt}{\isacharparenright}{\kern0pt}{\isacartoucheclose}\isanewline
\ \ \isakeyword{shows}\ {\isacartoucheopen}{\isasymP}{\isacharparenleft}{\kern0pt}R\ in\ run{\isacharunderscore}{\kern0pt}algo\ xs{\isachardot}{\kern0pt}\ {\isasymbar}R\ {\isacharminus}{\kern0pt}\ A{\isasymbar}\ {\isachargreater}{\kern0pt}\ {\isasymepsilon}\ {\isacharasterisk}{\kern0pt}\ A{\isacharparenright}{\kern0pt}\ {\isasymle}\ {\isasymdelta}{\isacartoucheclose}
\end{isabelle_cm}
The first line gives conditions on parameters $\varepsilon$ and $\delta$, which must be strictly between $0$ and $1$.
The next line requires the buffer size $n$ to be larger than or equal to $12 \varepsilon^{-2} \ln ( 3 \delta^{-1} l)$.
Then, we introduce the abbreviation \isa{A} for the cardinality of the set of elements in the sequence \isa{xs}.
%More precisely, \isa{set\ xs} denotes the finite set of distinct elements in a given sequence, and \isa{card} returns the cardinality of a finite set.
The notation \isa{\isasymP{\isacharparenleft}x\ in\ M{\isachardot}\ P\ x\isacharparenright} denotes the probability of a predicate $P$ in the probability space $M$, so the conclusion gives the PAC guarantee for the output estimate \isa{R} from \isa{run{\isacharunderscore}{\kern0pt}algo}.

Similarly, we have also formalized unbiasedness of \cref{alg:cvm_new}:
\begin{isabelle_cm}
\isacommand{theorem}\isamarkupfalse%
\ unbiasedness{\isacharcolon}{\kern0pt}\ {\isacartoucheopen}measure{\isacharunderscore}{\kern0pt}pmf{\isachardot}{\kern0pt}expectation\ {\isacharparenleft}{\kern0pt}run{\isacharunderscore}{\kern0pt}algo\ xs{\isacharparenright}{\kern0pt}\ id\ {\isacharequal}{\kern0pt}\ card\ {\isacharparenleft}{\kern0pt}set\ xs{\isacharparenright}{\kern0pt}{\isacartoucheclose}
\end{isabelle_cm}
where the expression \isa{measure{\isacharunderscore}pmf{\isachardot}expectation\ M\ f} denotes the expectation of the random variable \isa{f} on the probability space \isa{M}.

Our proofs are available both in mechanized form in Isabelle/HOL and as a pen-and-paper proof included in the associated proof document.
In practice, we developed the latter proof first and then mechanized it in Isabelle/HOL without much surprises.
Most of the lemmas can be one-to-one identified between both proofs; Isabelle's existing libraries, automation capabilities, and structured proof format were used extensively in our proofs.

%For our proof, we first sketched out our invariant-based approach in an informal proof before transferring this to Isabelle/HOL.
%Originally, we wrote down the informal proof first before starting the formalization effort.
%We could express the same arguments and same general strategy in Isabelle/HOL with little effort; we did not encounter any surprises during mechanization.
%This is in contrast to the mechanization of the original transformation-based proof by Chakraborty et al., which we detail in \cref{sec:transformation_based_proof}.

%The informal proof of the algorithm, which is included in the appendix of the formal proof document, is useful to follow the formalized proof.
%There are references from the formalized lemmas to the informal ones for convenience.

