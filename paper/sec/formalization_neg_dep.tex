\section{Formalization of the concept of negative association}\label{sec:formalization_neg_dep}
As mentioned in the previous section, the formalization of the new unbiased variant of the CVM algorithm, requires results from the theory of negative association.
The supplementary material contains a formalization of the concept in the directory Negative{\textunderscore}Association, which we intend to submit to the AFP as a separate contribution.

The formalization follows the definitions by Joag-Dev and Proschan~\cite{joagdev1983} closely.
However, the authors did not correctly define the constrains for the possible set of test functions $f$ and $g$ in \cref{def:neg_assoc}, which does not make sense formally.
In the formalization, we assume that the functions are bounded and measurable.
On the other hand, we derive that the condition is still true, if $f, g$ are square integrable.
Or alternatively integrable, and non-negative.
This is derived using the monotone-convergence theorem.

Another deviation from the original work is that we do not require that the random variables need to be real-valued.
In the formalization any linearly ordered topological space with the Borel $\sigma$-algebra is allowed as the range space.
In this case the test functions must be monotone with respect ot the respective order on that space.

The formalization contains the standard Chernoff bounds for negatively associated random variables.
This includes the additive bounds by Hoeffding~\cite[Th. 1, 2]{hoeffding1963} and the multipicative bounds by Motwani and Raghavan~\cite[Th. 4.1, 4.2]{motwani1995}.
Although, these results are not needed for the proof of the CVM algorithm.

A key issue, we faced during formalization, was the fact that there are many theorems of the form if a set of functions are either simultaneously monotone or anti-monotone.
For this we introduce a notation, that allows us to abstract over the direction of relations: \isa{\isasymle\isasymge\isactrlbsub\isasymeta\isactrlesub}.
Which evaluates to the forward version of the relation $\leq$ if \isa{\isasymeta\ \isacharequal\ Fwd} and the converse: $\geq$ if \isa{\isasymeta\ \isacharequal\ Rev}.
For example the FKG inequality
\[
  \expect [f g] \geq \expect [f] \expect [g]
\]
is true, if $f$ and $g$ are both montone, or both antimontone.
On the other hand the reverse inequality is true, if $f$ is montone and $g$ is antimontone, or vice versa.
Using our parameterized relation symbol, we could state all variants in a concise manner.


