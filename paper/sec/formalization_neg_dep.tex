\section{Formalization of a Library for Negative Association}\label{sec:formalization_neg_dep}
As mentioned in~\cref{sec:negdep}, formalizing the total and unbiased variant of the CVM algorithm requires results from the theory of negative association.

\begin{note}
In our supplementary material, the formalization of negative association is in the directory \verb|Neg_Assoc|.
This library contains key results used to establish the invariants for CVM (e.g., \verb|Neg_Assoc_Permutation_Distributions|).
Although not needed for CVM, we have also mechanized the standard Chernoff bounds (\verb|Neg_Assoc_Chernoff_Bounds|), including the additive bounds by Hoeffding~\cite[Th. 1, 2]{hoeffding1963} and the multiplicative bounds by Motwani and Raghavan~\cite[Th. 4.1, 4.2]{motwani1995}.
Another example application included in the library is proving the false positive rate of Bloom filters (\verb|Neg_Assoc_Bloom_Filters|).
In total the library contains 2974~lines of Isabelle code.
\lipicsEnd\end{note}

Our formalization follows the definitions by Joag-Dev and Proschan~\cite{joagdev1983} closely.
However, their definition leaves the class of test functions $f$ and $g$ (in \cref{def:neg_assoc}) imprecise.
In particular, for the formalization, we limit these functions to those that are bounded and measurable.
Additionally, we provide elimination rules, showing that if $X_1,\dots,X_n$ are negatively associated, then the inequality on expectations is still true even if $f, g$ are only square-integrable; or, alternatively, integrable and non-negative. This is derived using the monotone-convergence theorem.

Another deviation from the original work is that we do not require that the random variables are real-valued.
In the formalization, any linearly ordered topological space with the Borel $\sigma$-algebra is allowed as the range space.
In this case, the test functions must be monotone with respect to the respective order on the range space.

A key issue we faced during formalization, was that there are many theorems that condition on a set of functions being either simultaneously monotone or simultaneously anti-monotone.
To reduce duplication, we introduce a notation that allows us to abstract over the direction of relations: \isa{\isasymle\isasymge\isactrlbsub\isasymeta\isactrlesub}; it evaluates to the forward version of the relation $\leq$ if \isa{\isasymeta\ \isacharequal\ Fwd} and the converse: $\geq$ if \isa{\isasymeta\ \isacharequal\ Rev}.
For example the FKG inequality~\cite[Ch. 6]{alon2000},\cite{fortuin1971}
\[
  \expect [f g] \geq \expect [f] \expect [g]
\]
is true, if $f$ and $g$ are both monotone, or both antimonotone, on a probability space whose domain is a finite distributive lattice with a log-supermodular probability mass function.
The reverse inequality is also true, if $f$ is monotone and $g$ is antimonotone, or vice versa.
Using our parameterized relation symbol, we can state all variants in a concise manner:
\begin{isabelle_cm}
\isacommand{theorem}\isamarkupfalse%
\ fkg{\isacharunderscore}{\kern0pt}inequality{\isacharunderscore}{\kern0pt}pmf{\isacharcolon}{\kern0pt}\isanewline
\ \ \isakeyword{fixes}\ M\ {\isacharcolon}{\kern0pt}{\isacharcolon}{\kern0pt}\ {\isacartoucheopen}{\isacharparenleft}{\kern0pt}{\isacharprime}{\kern0pt}a\ {\isacharcolon}{\kern0pt}{\isacharcolon}{\kern0pt}\ finite{\isacharunderscore}{\kern0pt}distrib{\isacharunderscore}{\kern0pt}lattice{\isacharparenright}{\kern0pt}\ pmf{\isacartoucheclose}\isanewline
\ \ \isakeyword{fixes}\ f\ g\ {\isacharcolon}{\kern0pt}{\isacharcolon}{\kern0pt}\ {\isacartoucheopen}{\isacharprime}{\kern0pt}a\ {\isasymRightarrow}\ real{\isacartoucheclose}\isanewline
\ \ \isakeyword{assumes}\ {\isacartoucheopen}{\isasymAnd}x\ y{\isachardot}{\kern0pt}\ pmf\ M\ x\ {\isacharasterisk}{\kern0pt}\ pmf\ M\ y\ {\isasymle}\ pmf\ M\ {\isacharparenleft}{\kern0pt}x\ {\isasymsqunion}\ y{\isacharparenright}{\kern0pt}\ {\isacharasterisk}{\kern0pt}\ pmf\ M\ {\isacharparenleft}{\kern0pt}x\ {\isasymsqinter}\ y{\isacharparenright}{\kern0pt}{\isacartoucheclose}\isanewline
\ \ \isakeyword{assumes}\ {\isacartoucheopen}monotone\ {\isacharparenleft}{\kern0pt}{\isasymle}{\isacharparenright}{\kern0pt}\ {\isacharparenleft}{\kern0pt}{\isasymle}{\isasymge}\isactrlbsub {\isasymtau}\isactrlesub {\isacharparenright}{\kern0pt}\ f{\isacartoucheclose}\ {\isacartoucheopen}monotone\ {\isacharparenleft}{\kern0pt}{\isasymle}{\isacharparenright}{\kern0pt}\ {\isacharparenleft}{\kern0pt}{\isasymle}{\isasymge}\isactrlbsub {\isasymsigma}\isactrlesub {\isacharparenright}{\kern0pt}\ g{\isacartoucheclose}\isanewline
\ \ \isakeyword{shows}\ {\isacartoucheopen}{\isacharparenleft}{\kern0pt}{\isasymintegral}x{\isachardot}{\kern0pt}\ f\ x\ {\isasympartial}M{\isacharparenright}{\kern0pt}\ {\isacharasterisk}{\kern0pt}\ {\isacharparenleft}{\kern0pt}{\isasymintegral}x{\isachardot}{\kern0pt}\ g\ x\ {\isasympartial}M{\isacharparenright}{\kern0pt}\ {\isasymle}{\isasymge}\isactrlbsub {\isasymtau}\ {\isacharasterisk}{\kern0pt}\ {\isasymsigma}\isactrlesub \ {\isacharparenleft}{\kern0pt}{\isasymintegral}x{\isachardot}{\kern0pt}\ f\ x\ {\isacharasterisk}{\kern0pt}\ g\ x\ {\isasympartial}M{\isacharparenright}{\kern0pt}{\isacartoucheclose}
\end{isabelle_cm}
Here, \isa{\isasymsigma} and \isa{\isasymtau} are relation directions, and \isa{\isasymsigma\ \isacharasterisk\ \isasymtau} multiplies relation directions, i.e., \isa{\isasymsigma\ \isacharasterisk\ \isasymtau} is the forward direction if \isa{\isasymsigma} and \isa{\isasymtau} have the same direction, and it is the reverse direction otherwise.
The first assumption denotes the log-supermodularity of the probability mass function, while the second assumptions are the parametric monotonicity conditions.
The FKG inequality is a key result which enables verification of negative association for random variables.
This includes the indicator variables for the new subsampling operation we introduced in \cref{sec:negdep}.

Let us summarize a few key formalized results for negatively associated random variables in our library.
The following is the well-known Hoeffding inequality~\cite{hoeffding1963} generalized for negatively associated random variables.
\begin{isabelle_cm}
\isacommand{lemma}\isamarkupfalse%
\ hoeffding{\isacharunderscore}{\kern0pt}bound{\isacharunderscore}{\kern0pt}two{\isacharunderscore}{\kern0pt}sided{\isacharcolon}{\kern0pt}\isanewline
\ \ \isakeyword{assumes}\ {\isacartoucheopen}neg{\isacharunderscore}{\kern0pt}assoc\ X\ I{\isacartoucheclose}\ {\isacartoucheopen}finite\ I{\isacartoucheclose}\isanewline
\ \ \isakeyword{assumes}\ {\isacartoucheopen}{\isasymAnd}i{\isachardot}{\kern0pt}\ i{\isasymin}I\ {\isasymLongrightarrow}\ a\ i\ {\isasymle}\ b\ i{\isacartoucheclose}\isanewline
\ \ \isakeyword{assumes}\ {\isacartoucheopen}{\isasymAnd}i{\isachardot}{\kern0pt}\ i{\isasymin}I\ {\isasymLongrightarrow}\ AE\ {\isasymomega}\ in\ M{\isachardot}{\kern0pt}\ X\ i\ {\isasymomega}\ {\isasymin}\ {\isacharbraceleft}{\kern0pt}a\ i{\isachardot}{\kern0pt}{\isachardot}{\kern0pt}b\ i{\isacharbraceright}{\kern0pt}{\isacartoucheclose}\ {\isacartoucheopen}I\ {\isasymnoteq}\ {\isacharbraceleft}{\kern0pt}{\isacharbraceright}{\kern0pt}{\isacartoucheclose}\isanewline
\ \ \isakeyword{defines}\ {\isacartoucheopen}n\ {\isasymequiv}\ real\ {\isacharparenleft}{\kern0pt}card\ I{\isacharparenright}{\kern0pt}{\isacartoucheclose}\isanewline
\ \ \isakeyword{defines}\ {\isacartoucheopen}{\isasymmu}\ {\isasymequiv}\ {\isacharparenleft}{\kern0pt}{\isasymSum}i{\isasymin}I{\isachardot}{\kern0pt}\ expectation\ {\isacharparenleft}{\kern0pt}X\ i{\isacharparenright}{\kern0pt}{\isacharparenright}{\kern0pt}{\isacartoucheclose}\isanewline
\ \ \isakeyword{assumes}\ {\isacartoucheopen}{\isasymdelta}\ {\isasymge}\ {\isadigit{0}}{\isacartoucheclose}\ {\isacartoucheopen}{\isacharparenleft}{\kern0pt}{\isasymSum}i{\isasymin}I{\isachardot}{\kern0pt}\ {\isacharparenleft}{\kern0pt}b\ i{\isacharminus}{\kern0pt}a\ i{\isacharparenright}\isactrlsup {\isadigit{2}}{\isacharparenright}{\kern0pt}\ {\isachargreater}{\kern0pt}\ {\isadigit{0}}{\isacartoucheclose}\isanewline
\ \ \isakeyword{shows}\ {\isacartoucheopen}{\isasymP}{\isacharparenleft}{\kern0pt}{\isasymomega}\ in\ M{\isachardot}{\kern0pt}\ {\isasymbar}{\isacharparenleft}{\kern0pt}{\isasymSum}i{\isasymin}I{\isachardot}{\kern0pt}\ X\ i\ {\isasymomega}{\isacharparenright}{\kern0pt}{\isacharminus}{\kern0pt}{\isasymmu}{\isasymbar}\ {\isasymge}\ {\isasymdelta}{\isacharasterisk}{\kern0pt}n{\isacharparenright}{\kern0pt}\ {\isasymle}\ {\isadigit{2}}{\isacharasterisk}{\kern0pt}exp\ {\isacharparenleft}{\kern0pt}{\isacharminus}{\kern0pt}{\isadigit{2}}{\isacharasterisk}{\kern0pt}{\isacharparenleft}{\kern0pt}n{\isacharasterisk}{\kern0pt}{\isasymdelta}{\isacharparenright}{\kern0pt}\isactrlsup {\isadigit{2}}\ {\isacharslash}{\kern0pt}\ {\isacharparenleft}{\kern0pt}{\isasymSum}i{\isasymin}I{\isachardot}{\kern0pt}\ {\isacharparenleft}{\kern0pt}b\ i{\isacharminus}{\kern0pt}a\ i{\isacharparenright}\isactrlsup {\isadigit{2}}{\isacharparenright}{\kern0pt}{\isacharparenright}{\kern0pt}{\isacartoucheclose}
\end{isabelle_cm}
%\begin{isabelle_cm}
%\isacommand{lemma}\isamarkupfalse%
%\ neg{\isacharunderscore}{\kern0pt}assoc{\isacharunderscore}{\kern0pt}imp{\isacharunderscore}{\kern0pt}mult{\isacharunderscore}{\kern0pt}mono{\isacharcolon}{\kern0pt}\isanewline
%\ \ \isakeyword{fixes}\ f\ g\ {\isacharcolon}{\kern0pt}{\isacharcolon}{\kern0pt}\ {\isacartoucheopen}{\isacharparenleft}{\kern0pt}{\isacharprime}{\kern0pt}i\ {\isasymRightarrow}\ {\isacharprime}{\kern0pt}c{\isacharcolon}{\kern0pt}{\isacharcolon}{\kern0pt}linorder{\isacharunderscore}{\kern0pt}topology{\isacharparenright}{\kern0pt}\ {\isasymRightarrow}\ real{\isacartoucheclose}\isanewline
%\ \ \isakeyword{assumes}\ {\isacartoucheopen}neg{\isacharunderscore}{\kern0pt}assoc\ X\ I{\isacartoucheclose}\ {\isacartoucheopen}J\ {\isasymsubseteq}\ I{\isacartoucheclose}\isanewline
%\ \ \isakeyword{assumes}\ {\isacartoucheopen}square{\isacharunderscore}{\kern0pt}integrable\ M\ {\isacharparenleft}{\kern0pt}f\ {\isasymcirc}\ flip\ X{\isacharparenright}{\kern0pt}{\isacartoucheclose}\ {\isacartoucheopen}square{\isacharunderscore}{\kern0pt}integrable\ M\ {\isacharparenleft}{\kern0pt}g\ {\isasymcirc}\ flip\ X{\isacharparenright}{\kern0pt}{\isacartoucheclose}\isanewline
%\ \ \isakeyword{assumes}\ {\isacartoucheopen}monotone\ {\isacharparenleft}{\kern0pt}{\isasymle}{\isacharparenright}{\kern0pt}\ {\isacharparenleft}{\kern0pt}{\isasymle}{\isasymge}\isactrlbsub {\isasymeta}\isactrlesub {\isacharparenright}{\kern0pt}\ f{\isacartoucheclose}\ {\isacartoucheopen}monotone\ {\isacharparenleft}{\kern0pt}{\isasymle}{\isacharparenright}{\kern0pt}\ {\isacharparenleft}{\kern0pt}{\isasymle}{\isasymge}\isactrlbsub {\isasymeta}\isactrlesub {\isacharparenright}{\kern0pt}\ g{\isacartoucheclose}\isanewline
%\ \ \isakeyword{assumes}\ {\isacartoucheopen}depends{\isacharunderscore}{\kern0pt}on\ f\ J{\isacartoucheclose}\ {\isacartoucheopen}depends{\isacharunderscore}{\kern0pt}on\ g\ {\isacharparenleft}{\kern0pt}I{\isacharminus}{\kern0pt}J{\isacharparenright}{\kern0pt}{\isacartoucheclose}\isanewline
%\ \ \isakeyword{assumes}\ {\isacartoucheopen}f\ {\isasymin}\ borel{\isacharunderscore}{\kern0pt}measurable\ {\isacharparenleft}{\kern0pt}Pi\isactrlsub M\ J\ {\isacharparenleft}{\kern0pt}{\isasymlambda}{\isacharunderscore}{\kern0pt}{\isachardot}{\kern0pt}\ borel{\isacharparenright}{\kern0pt}{\isacharparenright}{\kern0pt}{\isacartoucheclose}\ \isanewline
%\ \ \isakeyword{assumes}\ {\isacartoucheopen}g\ {\isasymin}\ borel{\isacharunderscore}{\kern0pt}measurable\ {\isacharparenleft}{\kern0pt}Pi\isactrlsub M\ {\isacharparenleft}{\kern0pt}I{\isacharminus}{\kern0pt}J{\isacharparenright}{\kern0pt}\ {\isacharparenleft}{\kern0pt}{\isasymlambda}{\isacharunderscore}{\kern0pt}{\isachardot}{\kern0pt}\ borel{\isacharparenright}{\kern0pt}{\isacharparenright}{\kern0pt}{\isacartoucheclose}\isanewline
%\ \ \isakeyword{shows}\ {\isacartoucheopen}{\isacharparenleft}{\kern0pt}{\isasymintegral}{\kern0pt}{\isasymomega}{\isachardot}{\kern0pt}\ f{\isacharparenleft}{\kern0pt}{\isasymlambda}i{\isachardot}{\kern0pt}\ X\ i\ {\isasymomega}{\isacharparenright}{\kern0pt}{\isacharasterisk}{\kern0pt}g{\isacharparenleft}{\kern0pt}{\isasymlambda}i{\isachardot}{\kern0pt}\ X\ i\ {\isasymomega}{\isacharparenright}{\kern0pt}\ {\isasympartial}M{\isacharparenright}{\kern0pt}{\isasymle}{\isacharparenleft}{\kern0pt}{\isasymintegral}\isasymomega{\isachardot}{\kern0pt}\ f{\isacharparenleft}{\kern0pt}{\isasymlambda}i{\isachardot}{\kern0pt}\ X\ i\ \isasymomega{\isacharparenright}{\kern0pt}{\isasympartial}M{\isacharparenright}{\kern0pt}{\isacharasterisk}{\kern0pt}{\isacharparenleft}{\kern0pt}{\isasymintegral}{\kern0pt}\isasymomega{\isachardot}{\kern0pt}\ g{\isacharparenleft}{\kern0pt}{\isasymlambda}i{\isachardot}{\kern0pt}\ X\ i\ \isasymomega{\isacharparenright}{\kern0pt}{\isasympartial}M{\isacharparenright}{\kern0pt}{\isacartoucheclose}
%\end{isabelle_cm}
%This corresponds to Property~\ref{it:neg_dep_props:mult_mono} of \cref{pro:neg_dep_props}:
%If $X_i$ for $i \in I$ form a set of negatively associated variables, and $f$, $g$ are simultaneously monotone (or simultaneously antimonotone), real-valued, square-integrable, measurable functions, depending on disjoint components of $I$ then the expectations of the product are smaller than then product of the expectations of the composition of the functions $f,g$ with random variables $X_i$.
%Note that, if $X_i$ where independent the above would be true, without the monotonicity assumption.

Another key result (shown below) used for the verification of our CVM variant is the negative-associativity of the indicator functions of random $k$-subsets of a finite set $S$ (with cardinality greater than or equal to $k$).
\begin{isabelle_cm}
\isacommand{lemma}\isamarkupfalse%
\ n{\isacharunderscore}{\kern0pt}subsets{\isacharunderscore}{\kern0pt}distribution{\isacharunderscore}{\kern0pt}neg{\isacharunderscore}{\kern0pt}assoc{\isacharcolon}{\kern0pt}\isanewline
\ \ \isakeyword{assumes}\ {\isacartoucheopen}finite\ S{\isacartoucheclose}\ {\isacartoucheopen}k\ {\isasymle}\ card\ S{\isacartoucheclose}\isanewline
\ \ \isakeyword{defines}\ {\isacartoucheopen}p\ {\isasymequiv}\ pmf{\isacharunderscore}{\kern0pt}of{\isacharunderscore}{\kern0pt}set\ {\isacharbraceleft}{\kern0pt}T{\isachardot}{\kern0pt}\ T\ {\isasymsubseteq}\ S\ {\isasymand}\ card\ T\ {\isacharequal}{\kern0pt}\ k{\isacharbraceright}{\kern0pt}{\isacartoucheclose}\isanewline
\ \ \isakeyword{shows}\ {\isacartoucheopen}measure{\isacharunderscore}{\kern0pt}pmf{\isachardot}{\kern0pt}neg{\isacharunderscore}{\kern0pt}assoc\ p\ {\isacharparenleft}{\kern0pt}{\isasymin}{\isacharparenright}{\kern0pt}\ S{\isacartoucheclose}
\end{isabelle_cm}
This is a consequence of a more general result, which we have also shown, that permutation distributions are negatively associated.
We relied on the proof by Dubashi using the FKG inequality~\cite[Th. 10]{dubhashi1996}; there is a prior proof by Joag-Dev and Proschan~\cite[Th. 2.11]{joagdev1983}, which seemed to be incorrect.
%However Dubashi presented a correct proof later using the FKG inequality~\cite[Th. 10]{dubhashi1996}.}
%TODO: I suggest to put n_subsets_distribution_neg_assoc instead here instead, since the Hoeffding inequality is never used in our CVM proof.
%The following is the well-known Hoeffding inequality~\cite{hoeffding1963} for negatively associated random variables:
%\begin{isabelle_cm}
%\isacommand{lemma}\isamarkupfalse%
%\ hoeffding{\isacharunderscore}{\kern0pt}bound{\isacharunderscore}{\kern0pt}two{\isacharunderscore}{\kern0pt}sided{\isacharcolon}{\kern0pt}\isanewline
%\ \ \isakeyword{assumes}\ {\isacartoucheopen}neg{\isacharunderscore}{\kern0pt}assoc\ X\ I{\isacartoucheclose}\ {\isacartoucheopen}finite\ I{\isacartoucheclose}\isanewline
%\ \ \isakeyword{assumes}\ {\isacartoucheopen}{\isasymAnd}i{\isachardot}{\kern0pt}\ i{\isasymin}I\ {\isasymLongrightarrow}\ a\ i\ {\isasymle}\ b\ i{\isacartoucheclose}\isanewline
%\ \ \isakeyword{assumes}\ {\isacartoucheopen}{\isasymAnd}i{\isachardot}{\kern0pt}\ i{\isasymin}I\ {\isasymLongrightarrow}\ AE\ {\isasymomega}\ in\ M{\isachardot}{\kern0pt}\ X\ i\ {\isasymomega}\ {\isasymin}\ {\isacharbraceleft}{\kern0pt}a\ i{\isachardot}{\kern0pt}{\isachardot}{\kern0pt}b\ i{\isacharbraceright}{\kern0pt}{\isacartoucheclose}\ {\isacartoucheopen}I\ {\isasymnoteq}\ {\isacharbraceleft}{\kern0pt}{\isacharbraceright}{\kern0pt}{\isacartoucheclose}\isanewline
%\ \ \isakeyword{defines}\ {\isacartoucheopen}n\ {\isasymequiv}\ real\ {\isacharparenleft}{\kern0pt}card\ I{\isacharparenright}{\kern0pt}{\isacartoucheclose}\isanewline
%\ \ \isakeyword{defines}\ {\isacartoucheopen}{\isasymmu}\ {\isasymequiv}\ {\isacharparenleft}{\kern0pt}{\isasymSum}i{\isasymin}I{\isachardot}{\kern0pt}\ expectation\ {\isacharparenleft}{\kern0pt}X\ i{\isacharparenright}{\kern0pt}{\isacharparenright}{\kern0pt}{\isacartoucheclose}\isanewline
%\ \ \isakeyword{assumes}\ {\isacartoucheopen}{\isasymdelta}\ {\isasymge}\ {\isadigit{0}}{\isacartoucheclose}\ {\isacartoucheopen}{\isacharparenleft}{\kern0pt}{\isasymSum}i{\isasymin}I{\isachardot}{\kern0pt}\ {\isacharparenleft}{\kern0pt}b\ i{\isacharminus}{\kern0pt}a\ i{\isacharparenright}\isactrlsup {\isadigit{2}}{\isacharparenright}{\kern0pt}\ {\isachargreater}{\kern0pt}\ {\isadigit{0}}{\isacartoucheclose}\isanewline
%\ \ \isakeyword{shows}\ {\isacartoucheopen}{\isasymP}{\isacharparenleft}{\kern0pt}{\isasymomega}\ in\ M{\isachardot}{\kern0pt}\ {\isasymbar}{\isacharparenleft}{\kern0pt}{\isasymSum}i{\isasymin}I{\isachardot}{\kern0pt}\ X\ i\ {\isasymomega}{\isacharparenright}{\kern0pt}{\isacharminus}{\kern0pt}{\isasymmu}{\isasymbar}\ {\isasymge}\ {\isasymdelta}{\isacharasterisk}{\kern0pt}n{\isacharparenright}{\kern0pt}\ {\isasymle}\ {\isadigit{2}}{\isacharasterisk}{\kern0pt}exp\ {\isacharparenleft}{\kern0pt}{\isacharminus}{\kern0pt}{\isadigit{2}}{\isacharasterisk}{\kern0pt}{\isacharparenleft}{\kern0pt}n{\isacharasterisk}{\kern0pt}{\isasymdelta}{\isacharparenright}{\kern0pt}\isactrlsup {\isadigit{2}}\ {\isacharslash}{\kern0pt}\ {\isacharparenleft}{\kern0pt}{\isasymSum}i{\isasymin}I{\isachardot}{\kern0pt}\ {\isacharparenleft}{\kern0pt}b\ i{\isacharminus}{\kern0pt}a\ i{\isacharparenright}\isactrlsup {\isadigit{2}}{\isacharparenright}{\kern0pt}{\isacharparenright}{\kern0pt}{\isacartoucheclose}\end{isabelle_cm}

%TODO: I think the following footnote should be placed here, if we put negative association for permutation distribution here.

