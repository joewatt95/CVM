\section{Negative Dependence and an Unbiased CVM Algorithm\label{sec:negdep}}
Sometimes, sets of random variables possess a property called negative association.
The concept is a generalization of independence.
It has similar closure properties as independent random variables.
More interestingly, the standard Chernoff-Hoeffding bounds still apply to negatively associated random variables~\cite[Prop. 7]{dubhashi1998}.

The concept was introduced by Joag-Dev and Proschan~\cite{joagdev1983}, who showed that it has many useful closure properties compared to other previously introduced notions of negative dependence, such as negative correlation or negative orphant dependence. 

The following preliminary definition is useful, for the definition of the concept:
\begin{definition}
For a function defined on $n$-tuples $f: V^n \rightarrow W$, we will denote by $\mathrm{dep}(f)$ the set of coordinates the function depends on, e.g., $dep(f) \subseteq \{1,\ldots,n\}$ is minimal, s.t., $f(x) = f(y)$ for all $x, y \in V^n$ with $x_i = y_i$ for all $i \in dep(f)$.
\end{definition}

\begin{definition}[Negative Association]\label{def:neg_assoc}
A set of random variables $X_1,..,X_n: \Omega \rightarrow \mathbb R$ is negatively associated if for all non-decreasing functions $f,g: \mathbb R^n \rightarrow \mathbb R$, which depend on disjoint sets of the variables, i.e., $\mathrm{dep}(f) \cap \mathrm{dep}(g) = \emptyset$ then
\[
\expect [f(X_1,\ldots,X_n) g(X_1,\ldots,X_n)] \leq \expect [f(X_1,\ldots,X_n)] \expect [g(X_1,\ldots,X_n)] \textrm{.}
\]
\end{definition}

The following proposition summarizes some of the properties for negatively associated sets of random variables: 
\begin{proposition}\label{pro:neg_dep_props}
Summary of results for negatively associated random variables: 
\begin{enumerate}
\item If $X=(X_1,\ldots,X_n)$ are negatively associated then $\expect [f(X) g(X)] \leq \expect [f(X)] \expect [g(X)]$ also for non-increasing functions $f,g$ with $\mathrm{dep}(f) \cap \mathrm{dep}(g) = \emptyset$.
\item If $X=(X_1,\ldots,X_n)$ are negatively associated, $Y=(Y_1,\ldots,Y_m)$ are negatively associated, and the pair of vector valued random variables $X$ and $Y$ are independent, then the union $X_1,..,X_n,Y_1,..,Y_m$ is a set of negatively associated random variables.
\item If $X=(X_1,\ldots,X_n)$ are negatively associated, $f_1,..,f_m : \mathbb R^n \rightarrow \mathbb R$ be all non-increasing or all non-decreasing functions, s.t., $\mathrm{dep}(f_i) \cap \mathrm{dep}(f_j) = \emptyset$ for $i \neq j$, then $f_1(X),\ldots,f_m(X)$ form a set of negatively dependent random variables of size $m$.
\item If $X_1,..,X_n$ are independent then $X_1,...,X_n$ are negatively associated.
\item A subset of a negatively associated set of random variables is again negatively associated.
\end{enumerate}
\end{proposition}

These properties illustrate the trade-off between negative association and independence.
For example property 3 would be true for independent random variables, even without the condition of monotonicty.
Of course, on the other hand independence is a stronger property and fewer sets of random variables are indepenent.
For our discussion the following is an important lemma about negative associated random variables.

\begin{lemma}\label{le:neg_assoc_prod}
Let $X_1,..,X_n$ be negatively associated and $f_1,..,f_n$ be all non-decreasing or all non-increasing non-negative functions, then
\[
  \expect \left[\prod_{i=1}^{n} f_i(X_i)\right] \leq \prod_{i=1}^{n} f_i(\expect [X_i]) \textrm{.}
\]
\end{lemma}
\begin{proof}
Follows from the definition of negative associativy (or Property~1 of \cref{pro:neg_dep_props}, if the $f_i$ are non-increasing) using induction.
\end{proof}

The case for non-decreasing functions of the above lemma is pointed out by Joag-Dev and Proschan~\cite[P.2]{joagdev1983}.
Why we are interested in the lemma stems from the fact that indicator variables of random $m$-subsets are negatively associated.
The latter is a consequence of the fact that permutation distributions are negatively-associated~\cite[Th. 2.11]{joagdev1983}.%
\footnote{We could not verify the original proof by Joag-Dev and Proschan, which seems to be incorrect.
However Dubashi presented a correct proof later using the FKG inequality~\cite[Th. 10]{dubhashi1996}.}
Intuitively, this is interesting because it opens up an avenue to modifying the subsampling step in \cref{alg:cvm}.
Instead of keeping each element of $\chi$ with probability $\frac{1}{2}$, which has the disadvantage that there is a small possibility that none of the elements gets discarded, we could instead choose a random $nf$-subset, where $\frac{1}{2} \leq f < 1$, such that $nf$ is integer, in the subsampling step.
For such a subsampling step, we can derive using \cref{le:neg_assoc_prod}:
\begin{equation}\label{eq:subsample_with_n_subsets}
  \int_{\mathrm{subsample}(\chi)} \prod_{s \in S} g(I(s \in \chi)) \leq \prod_{s \in S} \int_{\mathrm{subsample}(\chi)} g(I(s \in \chi)) = \prod_{s \in S}  \expect_{\Ber(f)} g \textrm{.}
\end{equation}
for any non-negative $g$ and $S \subseteq \chi$. Note that $g$ is only evaluated at two points and is thus either non-increasing or non-decreasing.
Also, if $S$ is a singleton, the inequality becomes an equality.

Let us recall that this is the exact condition we needed for the subampling step (\cref{eq:subsample_condition}).
This means the proof for the modified algorithm in \cref{sec:invariants} works just as well with the new subsampling method.

The resulting algorithm is depicted in \cref{alg:cvm_new}.
\begin{algorithm}[h]
	\caption{New unbiased and total CVM algorithm.}\label{alg:cvm_new}
	\begin{algorithmic}[1]
  \Require Stream elements $a_1,\dots,a_l$, $0 < \varepsilon$, $0 < \delta < 1$.
  \Ensure A cardinality estimate $R$ for set $A = \{ a_1,\dots,a_l \}$ s.t. $\prob \left( |R - |A| | > \varepsilon |A| \right) \leq \delta$
  \State $\chi \gets \{\}, p \gets 1, n = \ceil*{\frac{12}{\varepsilon^2} \ln{(\frac{3l}{\delta})} }$, $\frac{1}{2} \leq f < 1$, s.t., $nf$ integer
  \For{$i \gets 1$ to $l$}
    \State $b \getsr \Ber(p)$ \Comment insert $a_i$ with probability $p$ (and remove it otherwise)
    \If{$b$}
      \State $\chi \gets \chi \cup \{a_i\}$
    \Else
      \State $\chi \gets \chi - \{a_i\}$
    \EndIf
    \If{$|\chi| = n$}
      \State $\chi \getsr \mathrm{subsample}(\chi)$ \Comment select a random $nf$ subset of $\chi$
      \State $p \gets pf$
    \EndIf
  \EndFor
  \State \Return $\frac{|\chi|}{p}$ \Comment estimate cardinality of $A$
  \end{algorithmic}
\end{algorithm}%
Note that it is possible to just choose $f = \frac{n-1}{n}$, i.e., discarding a random element from $\chi$ in the subsampling step.
The key advantage is that there is no line 11 and the algorithm never returns $\bot$.
And based on our discussion from the previous section it is also an unbiased algorithm, i.e., the expectation of the result is exactly $|A|$.

