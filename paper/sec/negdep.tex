\section{An Unbiased CVM Variant and Negative Dependence\label{sec:negdep}}

An interesting consequence of our invariant-based approach, is that it allowed us to devise and verify a refined version of the CVM algorithm, that is both total and unbiased.

\subsection{Unbiased CVM Variant}

\begin{algorithm}[t!]
	\caption{New unbiased and total CVM algorithm variant.}\label{alg:cvm_new}
	\begin{algorithmic}[1]
  \Require Stream elements $a_1,\dots,a_l$, $0 < \varepsilon$, $0 < \delta < 1$.
  \Ensure A cardinality estimate $R$ for set $A = \{ a_1,\dots,a_l \}$ s.t. $\prob \left( |R - |A| | > \varepsilon |A| \right) \leq \delta$
  \State $\chi \gets \{\}, p \gets 1, n = \ceil*{\frac{12}{\varepsilon^2} \ln{(\frac{3l}{\delta})} }$, $\frac{1}{2} \leq f < 1$, s.t., $nf$ integer
  \For{$i \gets 1$ to $l$}
    \State $b \getsr \Ber(p)$ \Comment insert $a_i$ with probability $p$ (and remove it otherwise)
    \If{$b$}
      \State $\chi \gets \chi \cup \{a_i\}$
    \Else
      \State $\chi \gets \chi - \{a_i\}$
    \EndIf
    \If{$|\chi| = n$} \Comment if buffer $\chi$ is full
      \State $\chi \getsr \mathrm{subsample}(\chi)$ \Comment select a random $nf$-subset of $\chi$
      \State $p \gets pf$
    \EndIf
  \EndFor
  \State \Return $\frac{|\chi|}{p}$ \Comment estimate cardinality of $A$
  \end{algorithmic}
\end{algorithm}%


When we look at the subsampling step of~\cref{alg:cvm}, our invariant~\cref{i:func_invariant} imposes the following condition on the subsampling operation:
\begin{equation}\label[ineq]{i:subsample_condition}
  \int_{\mathrm{subsample}(\chi)} \prod_{s \in S} g(\indicat(s \in \tau)) \, d \tau \leq \prod_{s \in S} \expect_{\Ber(f)} [g]
\end{equation}
for all non-negative functions $g$ and any $S \subseteq \chi$.
Any subsampling step that satisfies this functional inequality can be used while still preserving~\cref{i:func_invariant} for the algorithm.

Motivated by this, our new variant is shown in \cref{alg:cvm_new}.
For the subsampling step, instead of keeping each element of $\chi$ with probability $\frac{1}{2}$, we instead pick a uniform random $nf$-subset, where $\frac{1}{2} \leq f < 1$ and such that $nf$ is an integer.\footnote{It is possible to choose $f = \frac{n-1}{n}$, i.e., discarding a random element from $\chi$ in the subsampling step.}
Since this new subsampling step always reduces the size of $\chi$, the variant never returns $\bot$, i.e., it is \emph{total}.
The invariant-based approach allows us to show that the algorithm is probably-approximately correct and also \emph{unbiased}, i.e., the expectation of the result is exactly $|A|$.
These depend crucially on establishing~\cref{i:subsample_condition} for the new subsampler, for which we need a new concept.

\subsection{Background on Negative Dependence}
Some sets of random variables possess a property called \emph{negative association}, a generalization of independence.
The concept was introduced by Joag-Dev and Proschan~\cite{joagdev1983}, who showed that it has many useful closure properties compared to other previously introduced notions of negative dependence, such as negative correlation or negative orthant dependence. %TODO: perhaps citations for these two?
Importantly, standard Chernoff--Hoeffding type bounds still apply to negatively associated random variables~\cite[Prop. 7]{dubhashi1998}.
Negative association is defined as follows:
\begin{definition}
For a function defined on $n$-tuples $f: V^n \rightarrow W$, we will denote by $\mathrm{dep}(f)$ the set of coordinates the function depends on, i.e., $dep(f) \subseteq \{1,\ldots,n\}$ is minimal, s.t., $f(x) = f(y)$ for all $x, y \in V^n$ with $x_i = y_i$ for all $i \in dep(f)$.
\end{definition}

\begin{definition}[Negative Association]\label{def:neg_assoc}
A set of random variables $X_1,\dots,X_n: \Omega \rightarrow \mathbb R$ is negatively associated if, for all non-decreasing functions $f,g: \mathbb R^n \rightarrow \mathbb R$, which depend on disjoint sets of the variables, i.e., $\mathrm{dep}(f) \cap \mathrm{dep}(g) = \emptyset$, the following inequality holds.
\[
\expect [f(X_1,\ldots,X_n) g(X_1,\ldots,X_n)] \leq \expect [f(X_1,\ldots,X_n)] \expect [g(X_1,\ldots,X_n)] \textrm{.}
\]
\end{definition}

The following proposition summarizes some important properties of negatively associated sets of random variables.
\begin{proposition}[\cite{joagdev1983}]\label{pro:neg_dep_props}
Summary of results for negatively associated random variables.
\begin{enumerate}
\item \label{it:neg_dep_props:mult_mono} If $X=(X_1,\ldots,X_n)$ are negatively associated then $\expect [f(X) g(X)] \leq \expect [f(X)] \expect [g(X)]$ for non-increasing functions $f,g$ with $\mathrm{dep}(f) \cap \mathrm{dep}(g) = \emptyset$.
\item If $X=(X_1,\ldots,X_n)$ are negatively associated, $Y=(Y_1,\ldots,Y_m)$ are negatively associated, and the pair of vector-valued random variables $X$ and $Y$ are independent, then the union $X_1,\dots,X_n,Y_1,\dots,Y_m$ is a set of negatively associated random variables.
\item If $X=(X_1,\ldots,X_n)$ are negatively associated and $f_1, \dots ,f_m : \mathbb R^n \rightarrow \mathbb R$ are all non-increasing or all non-decreasing functions, s.t., $\mathrm{dep}(f_i) \cap \mathrm{dep}(f_j) = \emptyset$ for $i \neq j$, then $f_1(X),\ldots,f_m(X)$ form a set of negatively dependent random variables of size $m$.
\item If $X_1,\dots,X_n$ are independent then $X_1,\dots,X_n$ are negatively associated.
\item A subset of a negatively associated set of random variables is again negatively associated.
\end{enumerate}
\end{proposition}

These properties illustrate the trade-off between negative association and independence.
For example, Property 3 would be true for independent random variables, even without the condition of monotonicity.
%Of course, on the other hand independence is a stronger property and fewer sets of random variables are independent.
To analyze our new subsampler, the following is an important lemma about negative associated random variables.

\begin{lemma}\label{le:neg_assoc_prod}
Let $X_1,\dots,X_n$ be negatively associated and $f_1,\dots,f_n$ be all non-decreasing (or all non-increasing), non-negative functions, then
\[
  \expect \left[\prod_{i=1}^{n} f_i(X_i)\right] \leq \prod_{i=1}^{n} f_i(\expect [X_i]) \textrm{.}
\]
\end{lemma}
\begin{proof}
This follows from the definition of negative associativity (or Property~1 of \cref{pro:neg_dep_props}, if the $f_i$ are non-increasing) using induction.
\end{proof}

The case for non-decreasing functions of the above lemma is pointed out by Joag-Dev and Proschan~\cite[P.2]{joagdev1983}.
The reason for our interest in this lemma stems from the fact that indicator variables of random $m$-subsets are negatively associated.
This is a consequence of the fact that permutation distributions are negatively-associated~\cite[Th. 2.11]{joagdev1983}.
Thus, for the new subsampling step in Line 9 of~\cref{alg:cvm_new}, we can derive using \cref{le:neg_assoc_prod}:
\begin{equation}\label{eq:subsample_with_n_subsets}
  \int_{\mathrm{subsample}(\chi)} \prod_{s \in S} g(\indicat(s \in \tau))\,d\tau \leq
  \prod_{s \in S} \int_{\mathrm{subsample}(\chi)} g(\indicat(s \in \tau))\,d\tau = \prod_{s \in S}  \expect_{\Ber(f)}[g] \textrm{.}
\end{equation}
for any non-negative $g$ and $S \subseteq \chi$.
Note that the domain of $g$ has two values, so it is either non-increasing or non-decreasing.
Also, if $S$ is a singleton, the inequality becomes an equality.
With this ingredient, we can conclude that our results about the original algorithm we derived in the previous section also hold for our new alternative algorithm (\cref{alg:cvm_new}).
% Note: After reading this section. It stops too abrubtly at the previous sentence, without establishing what we have actually shown.
