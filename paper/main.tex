\RequirePackage[hyphens]{url}
\documentclass[a4paper,UKenglish,cleveref, autoref, thm-restate]{lipics-v2021}
%This is a template for producing LIPIcs articles.
%See lipics-v2021-authors-guidelines.pdf for further information.
%for A4 paper format use option "a4paper", for US-letter use option "letterpaper"
%for british hyphenation rules use option "UKenglish", for american hyphenation rules use option "USenglish"
%for section-numbered lemmas etc., use "numberwithinsect"
%for enabling cleveref support, use "cleveref"
%for enabling autoref support, use "autoref"
%for anonymousing the authors (e.g. for double-blind review), add "anonymous"
%for enabling thm-restate support, use "thm-restate"
%for enabling a two-column layout for the author/affilation part (only applicable for > 6 authors), use "authorcolumns"
%for producing a PDF according the PDF/A standard, add "pdfa"

%\pdfoutput=1 %uncomment to ensure pdflatex processing (mandatatory e.g. to submit to arXiv)
%\hideLIPIcs  %uncomment to remove references to LIPIcs series (logo, DOI, ...), e.g. when preparing a pre-final version to be uploaded to arXiv or another public repository

\usepackage{booktabs}
\usepackage{mdframed}
\usepackage{algorithm}
\usepackage[noend]{algpseudocode}

\renewcommand{\algorithmicrequire}{\textbf{Input:}}
\renewcommand{\algorithmicensure}{\textbf{Output:}}

\newcommand{\getsr}{\xleftarrow{\$}}
\newcommand{\todo}[1]{\textcolor{red}{#1}}
\newcommand{\ift}[3]{\mathbf{if} \; #1 \; \mathbf{then} \; #2 \; \mathbf{else} \; #3}
\newcommand{\integral}[3]{\int_{#1} \! #2 \, \mathrm{d} #3}
\newcommand{\isaprob}[3]{\isa{\isasymP{\isacharparenleft}#1 \isatext{in} #2{\isachardot} #3\isacharparenright}}

\usepackage{mathtools}
\DeclarePairedDelimiter{\ceil}{\lceil}{\rceil}
\DeclareMathOperator{\Rnonneg}{\mathbb R_{\geq 0}}
\DeclareMathOperator{\Ber}{\mathrm{Ber}}
\DeclareMathOperator{\prob}{\mathcal P}
\DeclareMathOperator{\expect}{\mathrm{E}}
\DeclareMathOperator{\indicat}{\mathrm{I}}
\DeclareMathOperator{\bigo}{\mathcal O}
%\graphicspath{{./graphics/}}%helpful if your graphic files are in another directory

\definecolor{shadecolor}{gray}{0.95}
\newenvironment{isabelle_cm}{\begin{mdframed}[backgroundcolor=shadecolor,nobreak=true,linewidth=0]\begin{isabelle}}{\end{isabelle}\end{mdframed}}%

\usepackage{isabelle}
\usepackage{isabellesym}
\isabellestyle{it}

\usepackage{tikz}
\usetikzlibrary{arrows.meta}
\usetikzlibrary{calc}
\usetikzlibrary{shapes.geometric}
\usetikzlibrary{decorations.pathreplacing,calligraphy}
\newcommand*\circled[1]{\tikz[baseline=(char.base)]{
            \node[shape=circle,draw, minimum size=3.5mm,inner sep=0.5pt] (char) {#1};}}

\crefname{ineq}{inequality}{inequalities}
\creflabelformat{ineq}{#2{\upshape(#1)}#3}

\newcommand\locnew{1003}
\newcommand\locold{2630}

\bibliographystyle{plainurl}% the mandatory bibstyle

%\title{Verification of the CVM algorithm with a New Recursive Analysis Technique} %TODO Please add
\title{Verification of the CVM algorithm with a Functional Probabilistic Invariant} %TODO maybe?

%\titlerunning{Dummy short title} %TODO optional, please use if title is longer than one line

% TODO: put in everyone's names here

\author{Yong Kiam Tan}{Institute for Infocomm Research (I$^2$R), A*STAR, Singapore \and Nanyang Technological University, Singapore}{yongkiam.tan@ntu.edu.sg}{https://orcid.org/0000-0001-7033-2463}{Singapore NRF Fellowship Programme NRF-NRFF16-2024-0002}
% \author{Seng Joe Watt}{Institute for Infocomm Research (I$^2$R), A*STAR, Singapore \and Nanyang Technological University, Singapore}{yongkiam.tan@ntu.edu.sg}{https://orcid.org/0000-0001-7033-2463}{Singapore NRF Fellowship Programme NRF-NRFF16-2024-0002}

\author{\textcolor{red}{Anonymous author(s)}}{\textcolor{red}{Anonymous affiliation(s)}}{}{}{}

\authorrunning{\textcolor{red}{Anonymous author(s)}} %TODO mandatory. First: Use abbreviated first/middle names. Second (only in severe cases): Use first author plus 'et al.'

\Copyright{\textcolor{red}{Anonymous author(s)}} %TODO mandatory, please use full first names. LIPIcs license is "CC-BY";  http://creativecommons.org/licenses/by/3.0/


\ccsdesc[500]{Theory of computation~Logic and verification}
\ccsdesc[500]{Theory of computation~Higher order logic}
\ccsdesc[500]{Mathematics of computing~Probabilistic algorithms}
\ccsdesc[500]{Theory of computation~Pseudorandomness and derandomization}

\keywords{Verification, Isabelle/HOL, Randomized Algorithms, Distinct Elements}

\category{} %optional, e.g. invited paper

\relatedversion{} %optional, e.g. full version hosted on arXiv, HAL, or other respository/website
%\relatedversiondetails[linktext={opt. text shown instead of the URL}, cite=DBLP:books/mk/GrayR93]{Classification (e.g. Full Version, Extended Version, Previous Version}{URL to related version} %linktext and cite are optional

\supplement{TODO}%optional, e.g. related research data, source code, ... hosted on a repository like zenodo, figshare, GitHub, ...
% AFP entry
\supplementdetails[linktext={opt. text shown instead of the URL}, cite=DBLP:books/mk/GrayR93, subcategory={Description, Subcategory}, swhid={Software Heritage Identifier}]{General Classification (e.g. Software, Dataset, Model, ...)}{URL to related version} %linktext, cite, and subcategory are optional
% Zenodo code repo
\supplementdetails[linktext={opt. text shown instead of the URL}, cite=DBLP:books/mk/GrayR93, subcategory={Description, Subcategory}, swhid={Software Heritage Identifier}]{General Classification (e.g. Software, Dataset, Model, ...)}{URL to related version} %linktext, cite, and subcategory are optional

%\funding{(Optional) general funding statement \dots}%optional, to capture a funding statement, which applies to all authors. Please enter author specific funding statements as fifth argument of the \author macro.

%\acknowledgements{I want to thank \dots}%optional

%\nolinenumbers %uncomment to disable line numbering

%Editor-only macros:: begin (do not touch as author)%%%%%%%%%%%%%%%%%%%%%%%%%%%%%%%%%%
\EventEditors{John Q. Open and Joan R. Access}
\EventNoEds{2}
\EventLongTitle{42nd Conference on Very Important Topics (CVIT 2016)}
\EventShortTitle{CVIT 2016}
\EventAcronym{CVIT}
\EventYear{2016}
\EventDate{December 24--27, 2016}
\EventLocation{Little Whinging, United Kingdom}
\EventLogo{}
\SeriesVolume{42}
\ArticleNo{23}
%%%%%%%%%%%%%%%%%%%%%%%%%%%%%%%%%%%%%%%%%%%%%%%%%%%%%%

\begin{document}

\maketitle

%NOTE: we have to inline all of these into one file for the final version if accepted.

\begin{abstract}
In 2022, Chakraborty et al.~\cite{chakraborty2022} published a streaming
algorithm (henceforth, the CVM algorithm) for the distinct
elements problem, that deviated considerably from the state-of-the art, due to its simplicity
and avoidance of standard derandomization techniques, while still maintaining a close to optimal
logarithmic space complexity.

In this entry, we verify the CVM algorithm's correctness using a new technique which simplifies
the formal analysis considerably compared to the orignal proof by Chakraborty et
al. The main idea is based on a probabilistic invariant that allows us to derive concentration bounds
using the Cram\'{e}r--Chernoff method.

This new technique opens up the possible algorithm design space, and we introduce a new variant of the
CVM algorithm, that is total, and also has an additional property in addition
to concentration: unbiasedness. This means the expected result of the algorithm is exactly equal to
the desired result. The latter is also a new property, that neither the original CVM algorithm
nor classic algorithms for the distinct elements problem possess.
\end{abstract}




\section{Introduction}
\label{sec:intro}

In 2022, Chakraborty, Vinodchandran, and Meel~\cite{chakraborty2022} published a marvelous streaming algorithm for the distinct elements problem which was very unexpected in the community~\cite{quanta}.
Indeed, Knuth later wrote a note on the algorithm~\cite{knuthnote}, pointing out its interesting properties and christening it the \emph{CVM} algorithm (which we use for the rest of this paper).
One striking property of the CVM algorithm is that, in contrast to every other known algorithm for the problem, it does not rely on hashing the stream elements.
Instead, the algorithm could theoretically be implemented in a setting where objects in the data stream only allow for equality comparisons.

Another property of CVM is its simplicity, both in terms of its description---the algorithm is listed in its entirety in~\cref{alg:cvm}---and its pen-and-paper correctness proof, which only requires undergraduate-level exposure to randomized algorithms.
This is why its authors called it ``an algorithm for the text book''.
The motivation of this work is to explore whether CVM's simplicity transfers into a mechanization within the Isabelle proof assistant~\cite{nipkow2002}.

\begin{algorithm}[h!]
	\caption{CVM algorithm for distinct elements estimation~\cite{chakraborty2022}.}\label{alg:cvm}
	\begin{algorithmic}[1]
  \Require Stream elements $a_1,\dots,a_l$, $0 < \varepsilon$, $0 < \delta < 1$.
  \Ensure A cardinality estimate $R$ for set $A = \{ a_1,\dots,a_l \}$ s.t. $\prob \left( |R - |A| | > \varepsilon |A| \right) \leq \delta$
  \State $\chi \gets \{\}, p \gets 1, n = \ceil*{\frac{12}{\varepsilon^2} \ln{(\frac{6l}{\delta})} }$
  \For{$i \gets 1$ to $l$}
    \State $b \getsr \Ber(p)$ \Comment insert $a_i$ with probability $p$ (and remove it otherwise)
    \If{$b$}
      \State $\chi \gets \chi \cup \{a_i\}$
    \Else
      \State $\chi \gets \chi - \{a_i\}$
    \EndIf
    \If{$|\chi| = n$}
      \State $\chi \getsr \mathrm{subsample}(\chi)$ \Comment discard elements of $\chi$ independently with prob. $\frac{1}{2}$
      \State $p \gets \frac{p}{2}$
    \EndIf
    \If{$|\chi| = n$}
      \Return $\bot$ \Comment fail if $\chi$ remains too large
    \EndIf
  \EndFor
  \State \Return $\frac{|\chi|}{p}$ \Comment estimate cardinality of $A$
  \end{algorithmic}
\end{algorithm}

Curiously, our formalization takes a substantially different route compared to the original proof~\cite{chakraborty2022,DBLP:journals/corr/abs-2301-10191}.
Briefly, the pen-and-paper analysis for CVM relies on a sequence of transformations of the algorithm.
The reason for these transformations is that standard methods for analyzing randomized algorithms, such as Chernoff--Hoeffding bounds, usually make statements about independent random variables.
Yet, for~\cref{alg:cvm} the random operations are far from being independent.\footnote{This was an incorrect claim in the initial published proof of CVM~\cite[Claim 6]{chakraborty2022}; a later version by the same authors~\cite{DBLP:journals/corr/abs-2301-10191} provides a correct proof.
The original error serves as a side motivation for this work.}
For example, in line 3 the Bernoulli distribution is sampled for a value $p$, which itself depends on previous random operations; similarly, the subsampling step in line 9 is only applied if the buffer is full, which also depends on previous random operations.
The aforementioned sequence of transformations bounds the error of~\cref{alg:cvm} in terms of another randomized algorithm with more desirable independence properties.
Indeed, it seems impossible to analyze~\cref{alg:cvm} directly using known methods.

This is the point where our new analysis technique comes in---it is very similar to how deterministic algorithms are verified using a loop invariant.
The key difference is that our choice of ``loop invariant'' for the randomized streaming algorithm is a functional probabilistic inequality, namely, we consider invariants of the form:
\[
  \expect [ h ] \leq h(c)
\]
where $h$ is allowed to range over a class of functions, and the expectation is taken over the distribution of the state of the algorithm after consuming each stream element.
By first establishing such an invariant for~\cref{alg:cvm}, we can then use it (via different choices of $h$) to establish error bounds for the algorithm.
For the rest of this paper, we explain this technique, its mechanization, and show how it leads to a simple but substantially more general proof of the CVM algorithm.
We believe the new proof remains accessible at the undergraduate level, albeit with some exposure to interactive theorem proving.

The main contributions of this work are:
\begin{itemize}
\item Introduction of a new technique using functional probabilistic invariants to verify randomized algorithms inductively/recursively.
\item Verification of the original CVM algorithm using our new technique.
\item Presentation and verification of a new variant of CVM that is total and unbiased.
\item Formalization of a theory of negatively associated random variables used to analyze the new CVM variant.
\end{itemize}

Verification of CVM using our new technique requires only 1044 lines in Isabelle~\cite{nipkow2002}, while the original proof required \todo{x} lines.
We detail some of the challenges faced when mechanizing the reduction arguments used by Chakraborty et al.~\cite{chakraborty2022,DBLP:journals/corr/abs-2301-10191} in Section~\todo{x}.

The formalized CVM variant showcases the utility of our new proof technique.
In this variant, the subsampling step in line 9 of \cref{alg:cvm} selects a random $m$-sized subset of $\chi$, instead of independently deciding for each element, whether to keep it.
This variant has the benefit that it is \emph{total} (never returns $\bot$) because the second check in line 11 becomes obsolete.
More interestingly, the resulting variant is \emph{unbiased}, i.e., the expected result is exactly the cardinality of the elements in the stream; this is a new property, that neither the original CVM algorithm nor classic algorithms for the distinct elements problem possess.

To analyze the new variant, we use results from the theory of negatively dependent random variables to establish the desired functional invariant.
The concept of negative association is a generalization of independence; negatively associated variables observe closure properties and fulfill Chernoff--Hoeffding bounds similar to independent random variables.
It should be stressed that the theory of negatively associated RVs is orthogonal to our new technique, but the formalization of such a theory is also a contribution of this work.

% TODO: move it into the discussion of the variant, I think it is better once the algorithm is shown
%This is because indicator functions of $n$-subsets form negatively associated random variables (RV), even though they are not independent, with which we can modify the subsampling step to the form we described above.

% TODO: move it into explanation of the technique
%Note that we are able to establish tail bounds using these invariants, which has, so far, not been possible using a simple loop invariant for randomized algorithms.
%The loop invariants are established, essentially, using the key property of the Giry monad:
%\[
%  \expect_{m \isa{\isasymbind} f} [h] = \int_m \expect_{f (x)} [h] \, d x \textrm{.}
%\]
%Here $m \isa{\isasymbind} f$ denotes distribution of a randomized algorithm, which represents the sequential composition of $m$ with $f$.

%TODO: I think this should be later, we can give a forward reference
%% The algorithm's state is a buffer $\chi$  (initially empty) and a fraction $p > 0$ (initially set to $1$).
%% The buffer contains a subset of the elements of the stream encountered so far, with maximal size $n$.
%% The size is chosen according to the desired accuracy parameters $\varepsilon$, $\delta$, and the stream size $l$.
%% The algorithm iterates over stream elements, adding each one to the buffer with probability $p$ or conversely -- if the current stream element is already in the buffer -- removing them with probability $(1-p)$.
%% If the buffer gets too large, approximately half of the elements are removed by discarding each element in $\chi$ independently with probability $\frac{1}{2}$; then, $p$ is adjusted to reflect the fact that the buffer now contains each element with probability $p_\text{new} = \frac{p_\text{old}}{2}$.
%% After processing the stream, the algorithm returns $\frac{|\chi|}{p}$ as an approximation of the number of distinct elements in the stream.
%% This output is probably-approximately correct, i.e., the probability that the relative error of $\frac{|\chi|}{p}$ exceeds $\varepsilon$ is at most $\delta$.
%The output of CVM is \emph{probably-approximately correct}, i.e., the probability that the relative error of its output exceeds $\varepsilon$ is at most $\delta$.
%Moreover, let us assume the space needed to store each element in the stream is $b$ bits, then the algorithm requires only $\bigo(\varepsilon^{-2} b \ln(\delta^{-1} l))$ bits of mutable state---far less than storing each stream element deterministically.%
%\footnote{The optimal randomized algorithm requires $\bigo( \varepsilon^{-2} \ln \delta + b)$ bits, but it requires more advanced algorithmic techniques. It would not be possible to present using such elementary steps as used in \cref{alg:cvm}, and involves computations in finite fields and random walks in expander graphs~\cite{blasiok2020, karayel2023}.}
%

\todo{Classic summary of sections.}


%% We set out to formalize the proof as described in the literature~\cite{chakraborty2022}, hoping to show that it would also be an easy formalization exercise.

%% %TODO: a bit too long in details, we should trim this and move to later section
%% \subparagraph*{The road not taken:}
%% One of the difficult aspects of the proof is that it relies on an eager-lazy coin flip conversion.
%% To understand that, we should note that none of the observable random variables, such as the presence of a stream element in the buffer or conditions on the value of $p$, are independent of the other state variables, which makes the algorithm hard to analyze and makes the application of standard techniques from probability theory, such as Hoeffding's theorem impossible.
%% The authors resolved that problem by a simulation argument---they show that~\cref{alg:cvm} behaves stochastically identically to a different algorithm, which makes the relevant coin flips in a different order.
%% That modified algorithm performs a column of coin flips for each stream element.
%% An element is kept in the buffer if the first $k=\log_2(p-1)$ rows of the column are heads.
%% At each sub-sampling step, when p is divided by two, i.e., if k increases by one.
%% The algorithm examines the newly activated $k$-th row of the previous sequence elements to decide whether the element should be kept in the buffer.
%% This preserves the invariant that the buffer consists of exactly those sequence elements whose associated coin flip column starts with $\log_2(p-1)$ heads.
%% Of course, the new algorithm is not practical for actual implementation, but one can verify its correctness using standard Chernoff bounds, and on the other hand, it is possible to show that its behavior is equivalent to Algorithm 1.
%% To summarize, while the algorithm is marvelous, the proof was still very technical.
%% The simulation argument, in particular, is not so elegant to formalize.

%% \subparagraph*{A more direct proof:}
%% We set out to try to find a more direct proof, which also eases the formalization effort.
%% For the following discussion, we will analyze Algorithm 1 with line 8 removed, i.e., the algorithm does not output $\bot$, nor performs a second check of $|\chi|=n$.
%% Note that this happens if the very improbable event where none of the elements in X are removed during a subsampling step, which happens with probability at most $2^{-n/2}$.
%% Overall, the probability of it happening during the course of the algorithm is at most $\frac{\delta}{2}$. (Note that removing line 8 does not affect the correctness of the algorithm, but it loses its space consumption bound.)
%% It is easy to see that any probability established about the algorithm missing line 8 will be true for the original algorithm with a possible correction by, at most,  $\frac{\delta}{2}$.
%% We will remember and correct this at the end of this section.

%% Let us consider an imaginary situation where, somehow, $p$ is fixed at some point in the algorithm.
%% For example, we could imagine a final sub-sampling loop, which is run until a fixed $p$ is reached.
%% Then, the indicator random variables representing the presence of a stream element TODO


\section{Background}
\label{sec:background}

\subsection{Randomized Algorithms and Distinct Elements}

The CVM algorithm is a \emph{streaming} algorithm for the distinct elements problem.
As shown in~\cref{alg:cvm}, given a data stream $a_1,\dots, a_l$, the goal of such algorithms is to return an accurate cardinality estimate for $A = \{a_1,\dots,a_l\}$.

Importantly, CVM is a \emph{probably-approximately correct} (PAC) algorithm where its output estimate $R$ satisfies
$\prob \left( |R - |A| | > \varepsilon |A| \right) \leq \delta$
for parameters $\varepsilon$ and $\delta$,
i.e., the probability that the relative error of $R$ exceeds $\varepsilon$ is at most $\delta$. % I think here it's good to have both a text and formal description.
Moreover, let us assume the space needed to store each element in the stream is $b$ bits, then the CVM algorithm requires only $\bigo(\varepsilon^{-2} b \ln(\delta^{-1} l))$ bits of mutable state, which is far less than storing each stream element deterministically.\footnote{The optimal randomized algorithm for distinct elements requires $\bigo( \varepsilon^{-2} \ln \delta + b)$ bits, but it requires more advanced algorithmic techniques. It would not be possible to present using such elementary steps as in \cref{alg:cvm} as it involves computations in finite fields and random walks in expander graphs~\cite{blasiok2020, karayel2023}.}

\subsection{Semantics of Randomized Algorithms}
We briefly review how reasoning about randomized algorithms works in Isabelle using the Giry monad~\cite{giry1982}.
A thorough discussion of the concept in the context of Isabelle has been written, for example, by Eberl et al.~\cite{eberl2020} and Lochbihler~\cite{lochbihler2016}.

The key idea is to model a randomized algorithm as a probability space representing the distribution of its results.
Let us consider \cref{alg:example}.
\begin{algorithm}[h!]
\caption{Example for sequential composition.}\label{alg:example}
\begin{algorithmic}[1]
\State $p \getsr \Ber(\frac{1}{2})$
\State $q \getsr \Ber(\frac{1}{3}+\frac{p}{2})$
\State \Return $q$
\end{algorithmic}
\end{algorithm}%

In the first step,~\cref{alg:example} flips a fair coin, such that $p$ is $1$ with probability $\frac{1}{2}$ and $0$ otherwise; the notation $\Ber(p)$ represents the Bernoulli distribution.
In the second step, the algorithm flips a coin $q$ which depends on $p$.
This has the consequence that we have to consider probability space-valued functions, like: $p \mapsto \Ber(\frac{1}{3}+\frac{p}{2})$, which is being \emph{bound} to the distribution of $p$.
The resulting distribution for $q$ is a \emph{compound distribution}, a combination of $\Ber(\frac{1}{3})$ and $\Ber(\frac{2}{3})$.

This example captures the main aspects of modeling randomized algorithms in the Giry monad.
Indeed, randomized algorithms can be modeled using the following ingredients:

\begin{description}
\item[Primitive Random Operations.] For example, a simple fair coin flip is represented using the Bernoulli distribution, $\Ber(\frac{1}{2})$.
\item[Return Combinator.]
Given a singleton element $x$, we can construct the singleton probability space, assigning probability $1$ to $x$ and $0$ to everything else.
In monad notation, this is written as: $\mathrm{\bf return}\, x$.

\item[Bind Combinator.]
The bind combinator represents sequential composition of two randomized algorithms $m$ and $f$, where the second randomized algorithm consumes the output of the first; in monad notation, this is: $m \isa{\isasymbind} f$.
Mathematically, this is the most involved operation, because $f$ is a function returning probability spaces, which takes inputs from the probability space $m$.

Let us consider an event $A$ in the probability space $m \isa{\isasymbind} f$.
Its probability can be evaluated by integrating over its probabilities in $f$ with respect to $m$:
\[
  \prob_{m \isa{\isasymbind} f} (A) = \int_m \prob_{f(x)} (A) \, d x \textrm{.}
\]

Another key property is the calculation of expectations;
if $h$ is a random variable over $m \isa{\isasymbind} f$, we can compute its expectation as:
\begin{equation}
  \label{eq:integral_bind}
  \expect_{m \isa{\isasymbind} f} [h] = \int_m \expect_{f(x)} [h] \, d x \textrm{.}
\end{equation}

\cref{eq:integral_bind} is crucially used to establish invariants of the form we will use in~\cref{sec:invariants}.
\end{description}
%\todo{I think we should briefly say that a streaming algorithm is modeled as folding over the monad?}


\section{Functional Probabilistic Invariants}\label{sec:invariants}
In this section, we will derive our new technique using \cref{alg:cvm} as an example.
Let us start by briefly reviewing the algorithm---its state is a buffer $\chi$ (initially empty) and a fraction $p > 0$ (initially set to $1$).
The buffer tracks a subset of the elements of the stream encountered so far, with maximal size $n$ chosen according to the desired accuracy parameters $\varepsilon$, $\delta$, and the stream size $l$.
The algorithm iterates over the stream elements, adding each one to the buffer with probability $p$ or conversely---if the current stream element is already in the buffer---removing it with probability $(1-p)$ (Lines 3--7).
If the number of elements in the buffer reaches the maximal size $n$, the subsampling operation is executed, which discards each element in $\chi$ independently with probability $\frac{1}{2}$; then, $p$ is adjusted to reflect the fact that the buffer now contains each element with probability $p_\text{new} = \frac{p_\text{old}}{2}$ (Lines 8--10).
If the subsampling operation fails, i.e., if no elements get discarded, then the algorithm fails returning $\bot$ (Line 11).
After processing the stream, the algorithm returns $\frac{|\chi|}{p}$ as a probably-approximately correct estimate for the number of distinct elements in the stream.

\begin{remark}
For our discussion below, it is convenient to analyze \cref{alg:cvm} without Line 11, i.e., we will skip the second check of $|\chi|=n$ determining whether the subsampling step succeeded.
This modified version simplifies our analysis as we do not have to worry about the possibility of the algorithm failing (returning $\bot$).
This transformation is also used in the original CVM proof~\cite{chakraborty2023}, where the total variational distance between these two variants of the algorithms is shown to be at most $\frac{\delta}{2}$.
This means probability bounds derived for the modified version, can be transferred to the original algorithm, with a correction term of $\frac{\delta}{2}$.
\end{remark}

\subsection{Deriving A Simple Probabilistic Invariant}
Let us consider the random variables $X_s := \indicat(s \in \chi)$ indicating the presence of a stream element $s \in A = \{a_1,\ldots,a_l\}$ in the buffer, where we write $\indicat$ for the indicator of a predicate, i.e., $\indicat(\mathrm{true}) = 1$ and $\indicat(\mathrm{false}) = 0$.
Before the algorithm first encounters the stream element $s$, $X_s$ will be $0$ unconditionally, because the buffer $\chi$ is always a subset of the stream elements processed so far, i.e., $\chi \subseteq \{a_1,\dots,a_m\}$ after loop iteration $m$.

In the loop iteration where element $s$ occurs the first time, it will be inserted with a probability $p$ in Lines 3--7.
This means, after Line 7, we have:
\begin{equation}
  \label{eq:indicator_eq}
  \expect [p^{-1} X_s] = 1 \textrm{.}
\end{equation}
Interestingly, this equation is preserved for the rest of the algorithm.
For example, let us consider a subsampling step: each $s$ is independently discarded with probability $\frac{1}{2}$ so $\prob(X_s=1)$ is halved, but so is $p$ after subsampling, which preserves the equation.

Let us explain how we can verify \cref{eq:indicator_eq} more formally.
For that, we model the state of the randomized algorithm as a pair $(\chi,p)$ and we write $\chi$ and $p$ for the random variables projecting their respective components from the distribution of the state of the algorithm.
%For that, let us describe the states of the algorithm using pairs composed of a subset of $A$ representing the buffer $\chi$ and a real number representing $p$.
%Note that, this means $p$ and $\chi$ are random variables, with respect to distributions of the state of the algorithm.
We will refer to parts of each loop iteration in \cref{alg:cvm} as $\mathrm{step}_1$ (resp.~$\mathrm{step}_2$) for Lines 3--7 (resp.~Lines 8--10).
The final distribution of the algorithm is the distribution resulting from the sequential composition of alternating steps:
\[
  \mathrm{init} \, \isa{\isasymbind}\; \mathrm{step}_1\, a_1 \; \isa{\isasymbind}\; \mathrm{step}_2 \; \isa{\isasymbind}\; \mathrm{step}_1 \, a_2 \; \isa{\isasymbind}\; \cdots \isa{\isasymbind}\; \mathrm{step}_1\, a_l \; \isa{\isasymbind}\; \mathrm{step}_2
\]
where we parameterize $\mathrm{step}_1$ with the stream element that it processes.
The term $\mathrm{init}$ represents the initial state, i.e., $\mathrm{init} = \mathrm{\bf return} (\{\},1)$.
It is easy to show by induction over the sequence of steps, we will have $0 < p \leq 1$ and $\chi \subseteq A$ for all possible states of the algorithm.

Let us verify that~\cref{eq:indicator_eq} is preserved as an invariant over all steps.
To verify that $\mathrm{step}_1\, a$ preserves \cref{eq:indicator_eq}, we assume some probability space of states $\Omega$ fulfills \cref{eq:indicator_eq} and we would like to show that it is still true for $\Omega \; \isa{\isasymbind}\; \mathrm{step}_1\, a$.
\[
  \expect_{\Omega \; \isa{\isasymbind}\; \mathrm{step}_1\, a} [ p^{-1} X_s ] =
    \int_\Omega \int_{\Ber(p)} p^{-1} \indicat\left(s \in (\ift{\tau}{\chi \cup \{a\}}{\chi-\{a\}} )\right) \, d \tau d \sigma
\]
Note that we write $p$ or $\chi$ even though, we should actually write $p(\sigma)$ or $\chi(\sigma)$, i.e., we remember that these implicitly depend on $\sigma$.
To see that the right-hand-side is equal to $1$, it is useful to consider the cases where $a=s$ and the converse separately.
When $a = s$, the right-hand-side is equal to $1$ by definition of the Bernoulli distribution (since $p \in (0;1]$).
When $a \not= s$, it follows from the induction hypothesis on $\Omega$; in particular, the term in the inner integral is constant with respect to $\tau$.

The same invariant-based argument is possible for $\mathrm{step}_2$, i.e., let us assume $\Omega$ is a probability space of states fulfilling \cref{eq:indicator_eq}.
Then
\[
  \expect_{\Omega \; \isa{\isasymbind}\; \mathrm{step_2}} \left[\frac{X_s}{p}\right] =
    \int_{\Omega} \left(\ift{|\chi|=n}{\left(\int_{\mathrm{subsample}(\chi)} \frac{\indicat(s \in \tau)}{p/2} \, d \tau\right)}{\frac{\indicat(s \in \chi)}{p}} \right) \, d \sigma \textrm{.}
\]
Note that the true- and false-cases of the inner if-then-else both evaluate to the same value: $p^{-1} \indicat(s \in \chi)$.
If $s \notin \chi$ both sides of the equation are $0$, because the subsampling operation returns a subset of $\chi$.
If $s \in \chi$ the probability that the element gets subsampled is $1/2$, so we arrive again at $\frac{1/2}{p/2} = p^{-1} \indicat(s \in \chi)$.
Hence: $\expect_{\Omega \; \isa{\isasymbind}\; \mathrm{step_2}} [p^{-1} X_s] = \expect_\Omega [p^{-1} X_s] = 1$.
This completes the invariance proof for~\cref{eq:indicator_eq}.

\subsection{Deriving A Functional Probabilistic Invariant}
With \cref{eq:indicator_eq} established, it is straightforward to show that the expected value of the output $p^{-1} |\chi|$ for the modified algorithm (without Line 11) is equal to the desired cardinality $|A|$.
However, recall that we are interested in verifying the algorithm's PAC guarantee.
A typical approach to establishing such a guarantee is to use Chernoff bounds which provide exponential tail bounds for the deviation of sums of independent random variables from their mean.
However, these are not directly useful in the CVM algorithm because the key random variables, e.g., $p^{-1} X_s$ for $s \in A$, are dependent.

An alternative is the Cram\'{e}r--Chernoff method, which is a general method to obtain tail bounds for any random variable.
It can be stated simply as $\prob(X \geq a) \leq M(t) e^{-ta}$ for all $t > 0$, where $M(t) := \expect [\exp(t X)]$ is the moment generating function of $X$.
%\footnote{The moment generating function is sometimes only defined for some values of $t$ (when the corresponding integral exists), and this is typically an interval including $0$; the inequality is only applicable wherever $M(t)$ is defined.}
%Since the probability space representing the state of the CVM algorithm is finite, $M(t)$ will always be defined in our case.
It is also possible to obtain lower tail bounds $\prob(X \leq a)$ using Cram\'{e}r--Chernoff method, which just requires estimates for $M(t)$ for $t < 0$, instead of $t > 0$.

In our case, we are interested in estimating the moment generating function of the random variable $p^{-1} |\chi|$ for the CVM algorithm:
\[
  \expect [\exp( t p^{-1} |\chi| )] = \expect \left[ \prod_{s \in A} h(p^{-1} X_s) \right]
\]
for $h(x) = \exp(tx)$.
At this point, it is tempting to see whether the proof for \cref{eq:indicator_eq} can be generalized to establish bounds for the above.
Indeed, we managed to establish the following result:
\begin{align}
  \expect \left[ \prod_{s \in A} h(p^{-1} X_s) \right] \leq h(1)^{|A|} \label[ineq]{i:func_invariant}
\end{align}
for every non-negative concave function $h : \mathbb R_{\geq 0} \rightarrow \mathbb R_{\geq 0}$.
We call this a \emph{functional probabilistic invariant} because one can establish it for all valid choices of $h$ with a single induction over steps of the randomized algorithm.

Of course, the exponential function in $M(t)$ is convex, so this is not yet the full story.
However, we can instead try to derive tail bounds for the random variable $\indicat(p \geq q) p^{-1} |\chi|$.
This leads to a similar inequality:
\begin{align}
  \expect \left[ \prod_{s \in A} I(p \geq q) h(p^{-1} X_s) \right] \leq h(1)^{|A|} \label[ineq]{i:func_invariant_capped}
\end{align}
with the new condition that $h$ needs to be non-negative and concave only on $[0;q^{-1}]$.
This then allows us to use approximate the exponential function from above with an affine function $h$ on the range $[0;q^{-1}]$, which yields tail bounds for $p^{-1} |\chi|$ under the condition $p \geq q$. As an example, the upper tail bound can be derived as follows:
%TODO: should we use exp(...) everywhere instead of mixing with e^ ? EK: That was the previous verison, main issue is the chain looks very lopsided, with almost all of the terms being on the right edge. (It looked too ugly.)--- OK
% The convention is now, if exp is applied to a random variable we use the long notation and otherwise the power notation, which might actually be (a bit) helpful.
\begin{eqnarray*}
  \prob( p^{-1} |\chi| \geq (1+\varepsilon) |A| \wedge p \geq q) & \leq & % Is not equal when |A|=0.
  \prob( \indicat(p \geq q) p^{-1} |\chi| \geq (1+\varepsilon) |A|) \\
  & \underset{\textrm{Markov}}{\leq} & e^{-t(1+\varepsilon) |A|} \expect \left[ \prod_{s \in A} \indicat(p \geq q) \exp( t p^{-1} X_s) \right] \\
  & \leq & e^{-t(1+\varepsilon) |A|} \expect \left[ \prod_{s \in A} \indicat(p \geq q) h(p^{-1} X_s) \right] \\
  & \underset{\textrm{Ineq.~\ref{i:func_invariant_capped}}}{\leq} & e^{-t(1+\varepsilon) |A|} h(1)^{|A|} \\
  & \underset{\textrm{Calculus}}{\leq} & e^{-n\varepsilon^2 / 12}
\end{eqnarray*}
where we choose $h(x) = 1+qx (e^{t/q}-1)$.
Note that $h$ is affine and it can be easily checked%
\footnote{Because the exponential function is convex and $h$ is affine, we only have to check the end points: $0, q^{-1}$.}
that it is an upper approximation of $\exp(tx)$ for $x \in [0;q^{-1}]$.
For the last step, we have to find the $t$ that produces the required bound.%
\footnote{We use $t = q \ln(1+\varepsilon)$ which is not the real optimum, but better for algebraic evaluation.}
To use these bounds, we also have to separately estimate $\prob(p < q)$.
For that, we use a similar strategy, as in the original proof by Chakraborty et al.~\cite{chakraborty2023}, with $q = \frac{n}{4 |A|}$.
%The formalization accompanying this work~\todo{[cite]} contains a detailed informal step-by-step proof using our approach in its appendix.
The formalization in the supplementary material contains a detailed informal step-by-step proof using our approach in its appendix.
Besides the use of \cref{eq:integral_bind} and the Cram\'er--Chernoff method, the steps are elementary.

The main takeaway here is that it is possible to derive useful and general probabilistic invariants by considering expectations of classes of functions of the state, proved using recursion or induction over the algorithm itself.
As far as we know this method of establishing tail bounds for randomized algorithms is new.


\section{An Unbiased CVM Variant and Negative Dependence\label{sec:negdep}}

An interesting consequence of our invariant-based approach, is that it allowed us to devise and verify a refined version of the CVM algorithm, that is both total and unbiased.

\subsection{Unbiased CVM Variant}

\begin{algorithm}[t!]
	\caption{New unbiased and total CVM algorithm variant.}\label{alg:cvm_new}
	\begin{algorithmic}[1]
  \Require Stream elements $a_1,\dots,a_l$, $0 < \varepsilon$, $0 < \delta < 1$.
  \Ensure A cardinality estimate $R$ for set $A = \{ a_1,\dots,a_l \}$ s.t. $\prob \left( |R - |A| | > \varepsilon |A| \right) \leq \delta$
  \State $\chi \gets \{\}, p \gets 1, n = \ceil*{\frac{12}{\varepsilon^2} \ln{(\frac{3l}{\delta})} }$, $\frac{1}{2} \leq f < 1$, s.t., $nf$ integer
  \For{$i \gets 1$ to $l$}
    \State $b \getsr \Ber(p)$ \Comment insert $a_i$ with probability $p$ (and remove it otherwise)
    \If{$b$}
      \State $\chi \gets \chi \cup \{a_i\}$
    \Else
      \State $\chi \gets \chi - \{a_i\}$
    \EndIf
    \If{$|\chi| = n$} \Comment if buffer $\chi$ is full
      \State $\chi \getsr \mathrm{subsample}(\chi)$ \Comment select a random $nf$-subset of $\chi$
      \State $p \gets pf$
    \EndIf
  \EndFor
  \State \Return $\frac{|\chi|}{p}$ \Comment estimate cardinality of $A$
  \end{algorithmic}
\end{algorithm}%


When we look at the subsampling step of~\cref{alg:cvm}, our invariant~\cref{i:func_invariant} imposes the following condition on the subsampling operation:
\begin{equation}\label[ineq]{i:subsample_condition}
  \int_{\mathrm{subsample}(\chi)} \prod_{s \in S} g(\indicat(s \in \tau)) \, d \tau \leq \prod_{s \in S} \expect_{\Ber(f)} [g]
\end{equation}
for all non-negative functions $g$ and any $S \subseteq \chi$---any subsampling step that satisfies this functional inequality can be used while still preserving~\cref{i:func_invariant} for the algorithm.

Motivated by this, our new variant is shown in \cref{alg:cvm_new}.
For the subsampling step, instead of keeping each element of $\chi$ with probability $\frac{1}{2}$, we instead pick a uniform random $nf$-subset, where $\frac{1}{2} \leq f < 1$ and such that $nf$ is an integer.\footnote{It is possible to choose $f = \frac{n-1}{n}$, i.e., discarding a random element from $\chi$ in the subsampling step.}
Since this new subsampling step always reduces the size of $\chi$, the variant never returns $\bot$, i.e., it is \emph{total}.
The invariant-based approach allows us to show that the algorithm is probably-approximately correct and also \emph{unbiased}, i.e., the expectation of the result is exactly $|A|$.
These depend crucially on establishing~\cref{i:subsample_condition} for the new subsampler, for which we need a new concept.

\subsection{Background on Negative Dependence}
Some sets of random variables possess a property called \emph{negative association}, a generalization of independence.
The concept was introduced by Joag-Dev and Proschan~\cite{joagdev1983}, who showed that it has many useful closure properties compared to other previously introduced notions of negative dependence, such as negative correlation or negative orthant dependence.
Importantly, standard Chernoff--Hoeffding type bounds still apply to negatively associated random variables~\cite[Prop. 7]{dubhashi1998}.
Negative association is defined as follows:
\begin{definition}
For a function defined on $n$-tuples $f: V^n \rightarrow W$, we will denote by $\mathrm{dep}(f)$ the set of coordinates the function depends on, i.e., $dep(f) \subseteq \{1,\ldots,n\}$ is minimal, s.t., $f(x) = f(y)$ for all $x, y \in V^n$ with $x_i = y_i$ for all $i \in dep(f)$.
\end{definition}

\begin{definition}[Negative Association]\label{def:neg_assoc}
A set of random variables $X_1,..,X_n: \Omega \rightarrow \mathbb R$ is negatively associated if, for all non-decreasing functions $f,g: \mathbb R^n \rightarrow \mathbb R$, which depend on disjoint sets of the variables, i.e., $\mathrm{dep}(f) \cap \mathrm{dep}(g) = \emptyset$, the following inequality holds.
\[
\expect [f(X_1,\ldots,X_n) g(X_1,\ldots,X_n)] \leq \expect [f(X_1,\ldots,X_n)] \expect [g(X_1,\ldots,X_n)] \textrm{.}
\]
\end{definition}

The following proposition summarizes some of the properties for negatively associated sets of random variables.
\begin{proposition}[\cite{joagdev1983}]\label{pro:neg_dep_props}
Summary of results for negatively associated random variables.
\begin{enumerate}
\item If $X=(X_1,\ldots,X_n)$ are negatively associated then $\expect [f(X) g(X)] \leq \expect [f(X)] \expect [g(X)]$ also for non-increasing functions $f,g$ with $\mathrm{dep}(f) \cap \mathrm{dep}(g) = \emptyset$.
\item If $X=(X_1,\ldots,X_n)$ are negatively associated, $Y=(Y_1,\ldots,Y_m)$ are negatively associated, and the pair of vector valued random variables $X$ and $Y$ are independent, then the union $X_1,..,X_n,Y_1,..,Y_m$ is a set of negatively associated random variables.
\item If $X=(X_1,\ldots,X_n)$ are negatively associated, $f_1,..,f_m : \mathbb R^n \rightarrow \mathbb R$ be all non-increasing or all non-decreasing functions, s.t., $\mathrm{dep}(f_i) \cap \mathrm{dep}(f_j) = \emptyset$ for $i \neq j$, then $f_1(X),\ldots,f_m(X)$ form a set of negatively dependent random variables of size $m$.
\item If $X_1,..,X_n$ are independent then $X_1,...,X_n$ are negatively associated.
\item A subset of a negatively associated set of random variables is again negatively associated.
\end{enumerate}
\end{proposition}

These properties illustrate the trade-off between negative association and independence.
For example, property 3 would be true for independent random variables, even without the condition of monotonicity.
%Of course, on the other hand independence is a stronger property and fewer sets of random variables are independent.
To analyze our new subsampler, the following is an important lemma about negative associated random variables.

\begin{lemma}\label{le:neg_assoc_prod}
Let $X_1,..,X_n$ be negatively associated and $f_1,..,f_n$ be all non-decreasing (or all non-increasing), non-negative functions, then
\[
  \expect \left[\prod_{i=1}^{n} f_i(X_i)\right] \leq \prod_{i=1}^{n} f_i(\expect [X_i]) \textrm{.}
\]
\end{lemma}
\begin{proof}
This follows from the definition of negative associativity (or Property~1 of \cref{pro:neg_dep_props}, if the $f_i$ are non-increasing) using induction.
\end{proof}

The case for non-decreasing functions of the above lemma is pointed out by Joag-Dev and Proschan~\cite[P.2]{joagdev1983}.
The reason for our interest in this lemma stems from the fact that indicator variables of random $m$-subsets are negatively associated.
This is a consequence of the fact that permutation distributions are negatively-associated~\cite[Th. 2.11]{joagdev1983}.%
\footnote{We could not verify the original proof by Joag-Dev and Proschan, which seems to be incorrect.
However Dubashi presented a correct proof later using the FKG inequality~\cite[Th. 10]{dubhashi1996}.}
Thus, for the new subsampling step in Line 9 of~\cref{alg:cvm_new}, we can derive using \cref{le:neg_assoc_prod}:
%TODO: double check!
\begin{equation}\label{eq:subsample_with_n_subsets}
  \int_{\mathrm{subsample}(\chi)} \prod_{s \in S} g(\indicat(s \in \tau))\,d\tau \leq
  \prod_{s \in S} \int_{\mathrm{subsample}(\chi)} g(\indicat(s \in \tau))\,d\tau = \prod_{s \in S}  \expect_{\Ber(f)}[g] \textrm{.}
\end{equation}
for any non-negative $g$ and $S \subseteq \chi$.
Note that the domain of $g$ has two values, so it is either non-increasing or non-decreasing.
Also, if $S$ is a singleton, the inequality becomes an equality.




\section{Formalization}
\todo{Overview of the two AFP entries. Include main statement of correctness for the CVM algorithm.}


\section{Formalization of a Library for Negative Association}\label{sec:formalization_neg_dep}
As mentioned in~\cref{sec:negdep}, formalizing the total and unbiased variant of the CVM algorithm requires results from the theory of negative association.

\begin{note}
The formalization of the theory of negative association is included in a separate AFP entry~\cite{Negative_Association-AFP}.
This library contains key results used to establish the invariants for CVM (e.g., \verb|Neg_Assoc_Permutation_Distributions|).
Although not needed for CVM, we have also mechanized the standard Chernoff bounds (\verb|Neg_Assoc_Chernoff_Bounds|), including the additive bounds by Hoeffding~\cite[Th. 1, 2]{hoeffding1963} and the multiplicative bounds by Motwani and Raghavan~\cite[Th. 4.1, 4.2]{motwani1995}.
Another example application included in the library is proving the false positive rate of Bloom filters (\verb|Neg_Assoc_Bloom_Filters|).
In total, the library contains 2974~lines of Isabelle code.
\lipicsEnd\end{note}

Our formalization follows the definitions by Joag-Dev and Proschan~\cite{joagdev1983} closely.
However, their definition leaves the class of test functions $f$ and $g$ (in \cref{def:neg_assoc}) imprecise.
In the formalization, we use different conditions for introduction and elimination rules.
In particular, for introduction rules, the test functions are bounded and measurable.
However, we provide stronger elimination rules, showing that if $X_1,\dots,X_n$ are negatively associated, then the inequality on expectations is true even if $f, g$ are only square-integrable; or, alternatively, integrable and non-negative. This is derived using the monotone convergence theorem.

Another deviation from the original work is that we do not require that the random variables are real-valued.
In the formalization, any linearly ordered topological space with the Borel $\sigma$-algebra is allowed as the range space.
In this case, the test functions must be monotone with respect to the respective order on the range space.

A key issue we faced during formalization was that there are many theorems that condition on a set of functions being either simultaneously monotone or simultaneously anti-monotone.
To reduce duplication, we introduce a notation that allows us to abstract over the direction of relations: \isa{\isasymle\isasymge\isactrlbsub\isasymeta\isactrlesub}; it evaluates to the forward version of the relation $\leq$ if \isa{\isasymeta\ \isacharequal\ Fwd} and the converse: $\geq$ if \isa{\isasymeta\ \isacharequal\ Rev}.
For example the FKG inequality~\cite[Ch. 6]{alon2000},\cite{fortuin1971}
\[
  \expect [f g] \geq \expect [f] \expect [g]
\]
is true, if $f$ and $g$ are both monotone, or both antimonotone, on a probability space whose domain is a finite distributive lattice with a log-supermodular probability mass function.
The reverse inequality is also true, if $f$ is monotone and $g$ is antimonotone, or vice versa.
Using our parameterized relation symbol, we can state all variants in a concise manner:
\begin{isabelle_cm}
\isacommand{theorem}\isamarkupfalse%
\ fkg{\isacharunderscore}{\kern0pt}inequality{\isacharunderscore}{\kern0pt}pmf{\isacharcolon}{\kern0pt}\isanewline
\ \ \isakeyword{fixes}\ M\ {\isacharcolon}{\kern0pt}{\isacharcolon}{\kern0pt}\ {\isacartoucheopen}{\isacharparenleft}{\kern0pt}{\isacharprime}{\kern0pt}a\ {\isacharcolon}{\kern0pt}{\isacharcolon}{\kern0pt}\ finite{\isacharunderscore}{\kern0pt}distrib{\isacharunderscore}{\kern0pt}lattice{\isacharparenright}{\kern0pt}\ pmf{\isacartoucheclose}\isanewline
\ \ \isakeyword{fixes}\ f\ g\ {\isacharcolon}{\kern0pt}{\isacharcolon}{\kern0pt}\ {\isacartoucheopen}{\isacharprime}{\kern0pt}a\ {\isasymRightarrow}\ real{\isacartoucheclose}\isanewline
\ \ \isakeyword{assumes}\ {\isacartoucheopen}{\isasymAnd}x\ y{\isachardot}{\kern0pt}\ pmf\ M\ x\ {\isacharasterisk}{\kern0pt}\ pmf\ M\ y\ {\isasymle}\ pmf\ M\ {\isacharparenleft}{\kern0pt}x\ {\isasymsqunion}\ y{\isacharparenright}{\kern0pt}\ {\isacharasterisk}{\kern0pt}\ pmf\ M\ {\isacharparenleft}{\kern0pt}x\ {\isasymsqinter}\ y{\isacharparenright}{\kern0pt}{\isacartoucheclose}\isanewline
\ \ \isakeyword{assumes}\ {\isacartoucheopen}monotone\ {\isacharparenleft}{\kern0pt}{\isasymle}{\isacharparenright}{\kern0pt}\ {\isacharparenleft}{\kern0pt}{\isasymle}{\isasymge}\isactrlbsub {\isasymtau}\isactrlesub {\isacharparenright}{\kern0pt}\ f{\isacartoucheclose}\ {\isacartoucheopen}monotone\ {\isacharparenleft}{\kern0pt}{\isasymle}{\isacharparenright}{\kern0pt}\ {\isacharparenleft}{\kern0pt}{\isasymle}{\isasymge}\isactrlbsub {\isasymsigma}\isactrlesub {\isacharparenright}{\kern0pt}\ g{\isacartoucheclose}\isanewline
\ \ \isakeyword{shows}\ {\isacartoucheopen}{\isacharparenleft}{\kern0pt}{\isasymintegral}x{\isachardot}{\kern0pt}\ f\ x\ {\isasympartial}M{\isacharparenright}{\kern0pt}\ {\isacharasterisk}{\kern0pt}\ {\isacharparenleft}{\kern0pt}{\isasymintegral}x{\isachardot}{\kern0pt}\ g\ x\ {\isasympartial}M{\isacharparenright}{\kern0pt}\ {\isasymle}{\isasymge}\isactrlbsub {\isasymtau}\ {\isacharasterisk}{\kern0pt}\ {\isasymsigma}\isactrlesub \ {\isacharparenleft}{\kern0pt}{\isasymintegral}x{\isachardot}{\kern0pt}\ f\ x\ {\isacharasterisk}{\kern0pt}\ g\ x\ {\isasympartial}M{\isacharparenright}{\kern0pt}{\isacartoucheclose}
\end{isabelle_cm}
Here, \isa{\isasymsigma} and \isa{\isasymtau} are relation directions, and \isa{\isasymsigma\ \isacharasterisk\ \isasymtau} multiplies relation directions, i.e., \isa{\isasymsigma\ \isacharasterisk\ \isasymtau} is the forward direction if \isa{\isasymsigma} and \isa{\isasymtau} have the same direction, and it is the reverse direction otherwise.
The first assumption denotes the log-supermodularity of the probability mass function, while the second assumptions are the parametric monotonicity conditions.
The FKG inequality is a key result which enables verification of negative association for random variables.
This includes the indicator variables for the new subsampling operation we introduced in \cref{sec:negdep}.

Let us summarize a few key formalized results for negatively associated random variables in our library.
The following is the well-known Hoeffding inequality~\cite{hoeffding1963} generalized for negatively associated random variables.
\begin{isabelle_cm}
\isacommand{lemma}\isamarkupfalse%
\ hoeffding{\isacharunderscore}{\kern0pt}bound{\isacharunderscore}{\kern0pt}two{\isacharunderscore}{\kern0pt}sided{\isacharcolon}{\kern0pt}\isanewline
\ \ \isakeyword{assumes}\ {\isacartoucheopen}neg{\isacharunderscore}{\kern0pt}assoc\ X\ I{\isacartoucheclose}\ {\isacartoucheopen}finite\ I{\isacartoucheclose}\isanewline
\ \ \isakeyword{assumes}\ {\isacartoucheopen}{\isasymAnd}i{\isachardot}{\kern0pt}\ i{\isasymin}I\ {\isasymLongrightarrow}\ a\ i\ {\isasymle}\ b\ i{\isacartoucheclose}\isanewline
\ \ \isakeyword{assumes}\ {\isacartoucheopen}{\isasymAnd}i{\isachardot}{\kern0pt}\ i{\isasymin}I\ {\isasymLongrightarrow}\ AE\ {\isasymomega}\ in\ M{\isachardot}{\kern0pt}\ X\ i\ {\isasymomega}\ {\isasymin}\ {\isacharbraceleft}{\kern0pt}a\ i{\isachardot}{\kern0pt}{\isachardot}{\kern0pt}b\ i{\isacharbraceright}{\kern0pt}{\isacartoucheclose}\ {\isacartoucheopen}I\ {\isasymnoteq}\ {\isacharbraceleft}{\kern0pt}{\isacharbraceright}{\kern0pt}{\isacartoucheclose}\isanewline
\ \ \isakeyword{defines}\ {\isacartoucheopen}n\ {\isasymequiv}\ real\ {\isacharparenleft}{\kern0pt}card\ I{\isacharparenright}{\kern0pt}{\isacartoucheclose}\isanewline
\ \ \isakeyword{defines}\ {\isacartoucheopen}{\isasymmu}\ {\isasymequiv}\ {\isacharparenleft}{\kern0pt}{\isasymSum}i{\isasymin}I{\isachardot}{\kern0pt}\ expectation\ {\isacharparenleft}{\kern0pt}X\ i{\isacharparenright}{\kern0pt}{\isacharparenright}{\kern0pt}{\isacartoucheclose}\isanewline
\ \ \isakeyword{assumes}\ {\isacartoucheopen}{\isasymdelta}\ {\isasymge}\ {\isadigit{0}}{\isacartoucheclose}\ {\isacartoucheopen}{\isacharparenleft}{\kern0pt}{\isasymSum}i{\isasymin}I{\isachardot}{\kern0pt}\ {\isacharparenleft}{\kern0pt}b\ i{\isacharminus}{\kern0pt}a\ i{\isacharparenright}\isactrlsup {\isadigit{2}}{\isacharparenright}{\kern0pt}\ {\isachargreater}{\kern0pt}\ {\isadigit{0}}{\isacartoucheclose}\isanewline
\ \ \isakeyword{shows}\ {\isacartoucheopen}{\isasymP}{\isacharparenleft}{\kern0pt}{\isasymomega}\ in\ M{\isachardot}{\kern0pt}\ {\isasymbar}{\isacharparenleft}{\kern0pt}{\isasymSum}i{\isasymin}I{\isachardot}{\kern0pt}\ X\ i\ {\isasymomega}{\isacharparenright}{\kern0pt}{\isacharminus}{\kern0pt}{\isasymmu}{\isasymbar}\ {\isasymge}\ {\isasymdelta}{\isacharasterisk}{\kern0pt}n{\isacharparenright}{\kern0pt}\ {\isasymle}\ {\isadigit{2}}{\isacharasterisk}{\kern0pt}exp\ {\isacharparenleft}{\kern0pt}{\isacharminus}{\kern0pt}{\isadigit{2}}{\isacharasterisk}{\kern0pt}{\isacharparenleft}{\kern0pt}n{\isacharasterisk}{\kern0pt}{\isasymdelta}{\isacharparenright}{\kern0pt}\isactrlsup {\isadigit{2}}\ {\isacharslash}{\kern0pt}\ {\isacharparenleft}{\kern0pt}{\isasymSum}i{\isasymin}I{\isachardot}{\kern0pt}\ {\isacharparenleft}{\kern0pt}b\ i{\isacharminus}{\kern0pt}a\ i{\isacharparenright}\isactrlsup {\isadigit{2}}{\isacharparenright}{\kern0pt}{\isacharparenright}{\kern0pt}{\isacartoucheclose}
\end{isabelle_cm}
%\begin{isabelle_cm}
%\isacommand{lemma}\isamarkupfalse%
%\ neg{\isacharunderscore}{\kern0pt}assoc{\isacharunderscore}{\kern0pt}imp{\isacharunderscore}{\kern0pt}mult{\isacharunderscore}{\kern0pt}mono{\isacharcolon}{\kern0pt}\isanewline
%\ \ \isakeyword{fixes}\ f\ g\ {\isacharcolon}{\kern0pt}{\isacharcolon}{\kern0pt}\ {\isacartoucheopen}{\isacharparenleft}{\kern0pt}{\isacharprime}{\kern0pt}i\ {\isasymRightarrow}\ {\isacharprime}{\kern0pt}c{\isacharcolon}{\kern0pt}{\isacharcolon}{\kern0pt}linorder{\isacharunderscore}{\kern0pt}topology{\isacharparenright}{\kern0pt}\ {\isasymRightarrow}\ real{\isacartoucheclose}\isanewline
%\ \ \isakeyword{assumes}\ {\isacartoucheopen}neg{\isacharunderscore}{\kern0pt}assoc\ X\ I{\isacartoucheclose}\ {\isacartoucheopen}J\ {\isasymsubseteq}\ I{\isacartoucheclose}\isanewline
%\ \ \isakeyword{assumes}\ {\isacartoucheopen}square{\isacharunderscore}{\kern0pt}integrable\ M\ {\isacharparenleft}{\kern0pt}f\ {\isasymcirc}\ flip\ X{\isacharparenright}{\kern0pt}{\isacartoucheclose}\ {\isacartoucheopen}square{\isacharunderscore}{\kern0pt}integrable\ M\ {\isacharparenleft}{\kern0pt}g\ {\isasymcirc}\ flip\ X{\isacharparenright}{\kern0pt}{\isacartoucheclose}\isanewline
%\ \ \isakeyword{assumes}\ {\isacartoucheopen}monotone\ {\isacharparenleft}{\kern0pt}{\isasymle}{\isacharparenright}{\kern0pt}\ {\isacharparenleft}{\kern0pt}{\isasymle}{\isasymge}\isactrlbsub {\isasymeta}\isactrlesub {\isacharparenright}{\kern0pt}\ f{\isacartoucheclose}\ {\isacartoucheopen}monotone\ {\isacharparenleft}{\kern0pt}{\isasymle}{\isacharparenright}{\kern0pt}\ {\isacharparenleft}{\kern0pt}{\isasymle}{\isasymge}\isactrlbsub {\isasymeta}\isactrlesub {\isacharparenright}{\kern0pt}\ g{\isacartoucheclose}\isanewline
%\ \ \isakeyword{assumes}\ {\isacartoucheopen}depends{\isacharunderscore}{\kern0pt}on\ f\ J{\isacartoucheclose}\ {\isacartoucheopen}depends{\isacharunderscore}{\kern0pt}on\ g\ {\isacharparenleft}{\kern0pt}I{\isacharminus}{\kern0pt}J{\isacharparenright}{\kern0pt}{\isacartoucheclose}\isanewline
%\ \ \isakeyword{assumes}\ {\isacartoucheopen}f\ {\isasymin}\ borel{\isacharunderscore}{\kern0pt}measurable\ {\isacharparenleft}{\kern0pt}Pi\isactrlsub M\ J\ {\isacharparenleft}{\kern0pt}{\isasymlambda}{\isacharunderscore}{\kern0pt}{\isachardot}{\kern0pt}\ borel{\isacharparenright}{\kern0pt}{\isacharparenright}{\kern0pt}{\isacartoucheclose}\ \isanewline
%\ \ \isakeyword{assumes}\ {\isacartoucheopen}g\ {\isasymin}\ borel{\isacharunderscore}{\kern0pt}measurable\ {\isacharparenleft}{\kern0pt}Pi\isactrlsub M\ {\isacharparenleft}{\kern0pt}I{\isacharminus}{\kern0pt}J{\isacharparenright}{\kern0pt}\ {\isacharparenleft}{\kern0pt}{\isasymlambda}{\isacharunderscore}{\kern0pt}{\isachardot}{\kern0pt}\ borel{\isacharparenright}{\kern0pt}{\isacharparenright}{\kern0pt}{\isacartoucheclose}\isanewline
%\ \ \isakeyword{shows}\ {\isacartoucheopen}{\isacharparenleft}{\kern0pt}{\isasymintegral}{\kern0pt}{\isasymomega}{\isachardot}{\kern0pt}\ f{\isacharparenleft}{\kern0pt}{\isasymlambda}i{\isachardot}{\kern0pt}\ X\ i\ {\isasymomega}{\isacharparenright}{\kern0pt}{\isacharasterisk}{\kern0pt}g{\isacharparenleft}{\kern0pt}{\isasymlambda}i{\isachardot}{\kern0pt}\ X\ i\ {\isasymomega}{\isacharparenright}{\kern0pt}\ {\isasympartial}M{\isacharparenright}{\kern0pt}{\isasymle}{\isacharparenleft}{\kern0pt}{\isasymintegral}\isasymomega{\isachardot}{\kern0pt}\ f{\isacharparenleft}{\kern0pt}{\isasymlambda}i{\isachardot}{\kern0pt}\ X\ i\ \isasymomega{\isacharparenright}{\kern0pt}{\isasympartial}M{\isacharparenright}{\kern0pt}{\isacharasterisk}{\kern0pt}{\isacharparenleft}{\kern0pt}{\isasymintegral}{\kern0pt}\isasymomega{\isachardot}{\kern0pt}\ g{\isacharparenleft}{\kern0pt}{\isasymlambda}i{\isachardot}{\kern0pt}\ X\ i\ \isasymomega{\isacharparenright}{\kern0pt}{\isasympartial}M{\isacharparenright}{\kern0pt}{\isacartoucheclose}
%\end{isabelle_cm}
%This corresponds to Property~\ref{it:neg_dep_props:mult_mono} of \cref{pro:neg_dep_props}:
%If $X_i$ for $i \in I$ form a set of negatively associated variables, and $f$, $g$ are simultaneously monotone (or simultaneously antimonotone), real-valued, square-integrable, measurable functions, depending on disjoint components of $I$ then the expectations of the product are smaller than then product of the expectations of the composition of the functions $f,g$ with random variables $X_i$.
%Note that, if $X_i$ where independent the above would be true, without the monotonicity assumption.

Another key result (shown below) used for the verification of our CVM variant is the negative-associativity of the indicator functions of random $k$-subsets of a finite set $S$ (with cardinality greater than or equal to $k$).
\begin{isabelle_cm}
\isacommand{lemma}\isamarkupfalse%
\ n{\isacharunderscore}{\kern0pt}subsets{\isacharunderscore}{\kern0pt}distribution{\isacharunderscore}{\kern0pt}neg{\isacharunderscore}{\kern0pt}assoc{\isacharcolon}{\kern0pt}\isanewline
\ \ \isakeyword{assumes}\ {\isacartoucheopen}finite\ S{\isacartoucheclose}\ {\isacartoucheopen}k\ {\isasymle}\ card\ S{\isacartoucheclose}\isanewline
\ \ \isakeyword{defines}\ {\isacartoucheopen}p\ {\isasymequiv}\ pmf{\isacharunderscore}{\kern0pt}of{\isacharunderscore}{\kern0pt}set\ {\isacharbraceleft}{\kern0pt}T{\isachardot}{\kern0pt}\ T\ {\isasymsubseteq}\ S\ {\isasymand}\ card\ T\ {\isacharequal}{\kern0pt}\ k{\isacharbraceright}{\kern0pt}{\isacartoucheclose}\isanewline
\ \ \isakeyword{shows}\ {\isacartoucheopen}measure{\isacharunderscore}{\kern0pt}pmf{\isachardot}{\kern0pt}neg{\isacharunderscore}{\kern0pt}assoc\ p\ {\isacharparenleft}{\kern0pt}{\isasymin}{\isacharparenright}{\kern0pt}\ S{\isacartoucheclose}
\end{isabelle_cm}
This is a consequence of a more general result, which we have also shown, that permutation distributions are negatively associated.
We relied on the proof by Dubhashi et al.\ using the FKG inequality~\cite[Th. 10]{dubhashi1996}; there is a prior proof by Joag-Dev and Proschan~\cite[Th. 2.11]{joagdev1983}, which is incomplete.\footnote{The step which we could not directly formalize is the assertion (Sentence~14) in the proof of Theorem~2.11 (\cite{joagdev1983}) that the conditional expectation of the random variable $f(X)$ is smaller iff the smallest element of the permutation is contained in $\mathrm{dep}(f)$. That statement is non-trivial and requires a proof using a theorem such as the FKG-inequality. We think Dubhashi et al.\ developed their proof to complete it.}
%However Dubhashi presented a correct proof later using the FKG inequality~\cite[Th. 10]{dubhashi1996}.}
%TODO: I suggest to put n_subsets_distribution_neg_assoc instead here instead, since the Hoeffding inequality is never used in our CVM proof.
%The following is the well-known Hoeffding inequality~\cite{hoeffding1963} for negatively associated random variables:
%\begin{isabelle_cm}
%\isacommand{lemma}\isamarkupfalse%
%\ hoeffding{\isacharunderscore}{\kern0pt}bound{\isacharunderscore}{\kern0pt}two{\isacharunderscore}{\kern0pt}sided{\isacharcolon}{\kern0pt}\isanewline
%\ \ \isakeyword{assumes}\ {\isacartoucheopen}neg{\isacharunderscore}{\kern0pt}assoc\ X\ I{\isacartoucheclose}\ {\isacartoucheopen}finite\ I{\isacartoucheclose}\isanewline
%\ \ \isakeyword{assumes}\ {\isacartoucheopen}{\isasymAnd}i{\isachardot}{\kern0pt}\ i{\isasymin}I\ {\isasymLongrightarrow}\ a\ i\ {\isasymle}\ b\ i{\isacartoucheclose}\isanewline
%\ \ \isakeyword{assumes}\ {\isacartoucheopen}{\isasymAnd}i{\isachardot}{\kern0pt}\ i{\isasymin}I\ {\isasymLongrightarrow}\ AE\ {\isasymomega}\ in\ M{\isachardot}{\kern0pt}\ X\ i\ {\isasymomega}\ {\isasymin}\ {\isacharbraceleft}{\kern0pt}a\ i{\isachardot}{\kern0pt}{\isachardot}{\kern0pt}b\ i{\isacharbraceright}{\kern0pt}{\isacartoucheclose}\ {\isacartoucheopen}I\ {\isasymnoteq}\ {\isacharbraceleft}{\kern0pt}{\isacharbraceright}{\kern0pt}{\isacartoucheclose}\isanewline
%\ \ \isakeyword{defines}\ {\isacartoucheopen}n\ {\isasymequiv}\ real\ {\isacharparenleft}{\kern0pt}card\ I{\isacharparenright}{\kern0pt}{\isacartoucheclose}\isanewline
%\ \ \isakeyword{defines}\ {\isacartoucheopen}{\isasymmu}\ {\isasymequiv}\ {\isacharparenleft}{\kern0pt}{\isasymSum}i{\isasymin}I{\isachardot}{\kern0pt}\ expectation\ {\isacharparenleft}{\kern0pt}X\ i{\isacharparenright}{\kern0pt}{\isacharparenright}{\kern0pt}{\isacartoucheclose}\isanewline
%\ \ \isakeyword{assumes}\ {\isacartoucheopen}{\isasymdelta}\ {\isasymge}\ {\isadigit{0}}{\isacartoucheclose}\ {\isacartoucheopen}{\isacharparenleft}{\kern0pt}{\isasymSum}i{\isasymin}I{\isachardot}{\kern0pt}\ {\isacharparenleft}{\kern0pt}b\ i{\isacharminus}{\kern0pt}a\ i{\isacharparenright}\isactrlsup {\isadigit{2}}{\isacharparenright}{\kern0pt}\ {\isachargreater}{\kern0pt}\ {\isadigit{0}}{\isacartoucheclose}\isanewline
%\ \ \isakeyword{shows}\ {\isacartoucheopen}{\isasymP}{\isacharparenleft}{\kern0pt}{\isasymomega}\ in\ M{\isachardot}{\kern0pt}\ {\isasymbar}{\isacharparenleft}{\kern0pt}{\isasymSum}i{\isasymin}I{\isachardot}{\kern0pt}\ X\ i\ {\isasymomega}{\isacharparenright}{\kern0pt}{\isacharminus}{\kern0pt}{\isasymmu}{\isasymbar}\ {\isasymge}\ {\isasymdelta}{\isacharasterisk}{\kern0pt}n{\isacharparenright}{\kern0pt}\ {\isasymle}\ {\isadigit{2}}{\isacharasterisk}{\kern0pt}exp\ {\isacharparenleft}{\kern0pt}{\isacharminus}{\kern0pt}{\isadigit{2}}{\isacharasterisk}{\kern0pt}{\isacharparenleft}{\kern0pt}n{\isacharasterisk}{\kern0pt}{\isasymdelta}{\isacharparenright}{\kern0pt}\isactrlsup {\isadigit{2}}\ {\isacharslash}{\kern0pt}\ {\isacharparenleft}{\kern0pt}{\isasymSum}i{\isasymin}I{\isachardot}{\kern0pt}\ {\isacharparenleft}{\kern0pt}b\ i{\isacharminus}{\kern0pt}a\ i{\isacharparenright}\isactrlsup {\isadigit{2}}{\isacharparenright}{\kern0pt}{\isacharparenright}{\kern0pt}{\isacartoucheclose}\end{isabelle_cm}

%TODO: I think the following footnote should be placed here, if we put negative association for permutation distribution here.



\section{Original Proof}
In this section we describe the interesting parts of the original proof by Chakraborty et al.~\cite{chakraborty2022}.
This highlights in part, why it requires a lot of work to formalize, and why we were motivated to develop our new approach using probabilistic invariants.

As we mentioned in the introduction, the main difficulty is the fact that \cref{alg:cvm}'s state variables are not independent.
The trick Chakraborty et al. use to circumvent the problem is by modifying the algorithm, in a manner that obviously preserves its distribution, such that the resulting algorithm's state can be described in terms of independent coin flips.

Let us consider a state, where $k$ subsampling steps have been performed, i.e., $p = 2^{-k}$. 
Then the algorithm would normally perform a coin flip with probability $p$.
In the modified algorithm, we perform a fixed number of coin flips for each sequence element at the beginning.
The element is put into the sample, whenever the first $k$ coin flips associated with the sequence element are $1$.
Note that this happens exactly with probability $2^{-k}$, which means the behaviour of the algorithm is unchanged.

Moreover, during the subsampling operation those elements are kept, whose $k+1$-th associated coin flip is $1$.
This again preserves the behaviour, that each element is discared independently with probability $1/2$.
The operation $p \mapsto \frac{p}{2}$ is replaced with $k \mapsto k+1$.
For this modified algorithm, it is easy to see that the set of elements in any state are exactly, those stream elements for which the first $k$ entries of their associated coin flips are $1$.

In \cref{alg:cvm_simul} we describe the new algorithm, that behaves exactly as the original one (\cref{alg:cvm}), but performs indepenent coin flips.
Note that the function $\mathrm{last{\textunderscore}index}$ returns the index of the last-occurence of an element in the sequence, before the current loop iteration.
It should be noted that the algorithm keeps track of the number of subsampling iterations $k$, instead of the value $p = 2^{-k}$ as the original algorithm does.
%Like, we did in \cref{sec:invariants}, we ignore the second check, whether the subsampling operation succeeded.
%As we explained there, the total variational difference between these two variants is $\frac{\delta}{2}$.
%
\begin{algorithm}[h!]
	\caption{Modified CVM algorithm with independent coin flips.}\label{alg:cvm_simul}
	\begin{algorithmic}[1]
  \Require Stream elements $a_1,\dots,a_l$, $0 < \varepsilon$, $0 < \delta < 1$.
  \State $\chi \gets \{\}, k \gets 0, n = \ceil*{\frac{12}{\varepsilon^2} \ln{(\frac{6l}{\delta})} }$
  \State $b[i,j] \getsr \Ber(1/2)$ for $i,j \in \{1,\cdots,l\}$ \Comment perform $l^2$ unbiased independent coin flips
  \For{$i \gets 1$ to $l$}
    \If{$b[i,1]=b[i,2]=\cdots=b[i,k]=1$}
      \State $\chi \gets \chi \cup \{a_i\}$
    \Else
      \State $\chi \gets \chi - \{a_i\}$
    \EndIf
    \If{$|\chi| = n$}
      \State $\chi \gets \{a \in \chi | b[\mathrm{last{\textunderscore}index}(a),k+1] = 1\}$
      \State $k \gets k+1$
    \EndIf
    \If{$|\chi| = n$}
      \State \Return $\bot$
    \EndIf
  \EndFor
  \State \Return $2^k |\chi|$ \Comment estimate cardinality of $A$
  \end{algorithmic}
\end{algorithm}
To roughly explain, how tail bounds can be derived for this new variant:
It is possible to union bound the probability that the estimate exceeds the desired interval, with the probability of the event in conjunction with a specific value of $k$, which can be bounded by the probability that the number of stream elements with whose associated coin flips start with $k$ ones, is outside of $2^{-k} |A| (1 \pm \varepsilon)$.
This is explained in more detail by Chakraborty et al.~\cite{chakraborty2022}.

One of the key questions is, how to formalize the transformation from \cref{alg:cvm} to this new variant.
What we discovered is that it best to solve the problem backwards, i.e., we start with the modified algorithm, which performs all the coin-flips in advance --- eagerly --- and convert it to the version, that performs the coin flips --- lazily --- at the point they are needed.

We can push down the coin flips through the expression tree.
To explain how this works.
Let us first define the \emph{sampling} function, i.e., let $f$ be a function that depends on coin flips, more precisely $f$ takes as argument a vector of coin flips indexed by $I$, then we can express the distribution of $f$ with respect to indepenent unbiased coin flips as.
\begin{isabelle_cm}
  sample\ f\ \isacharequal\ map{\isacharunderscore}pmf\ f\ {\isacharparenleft}prod{\isacharunderscore}pmf\ I\ {\isacharparenleft}\isasymlambda\isacharunderscore\isachardot\ bernoulli{\isacharunderscore}pmf \isacharparenleft\isadigit{1}/\isadigit{2}\isacharparenright\isacharparenright\isacharparenright
\end{isabelle_cm}
%For example, we could use our modified algorithm as $f$, with the index set $I = \{0, \ldots, l-1\} \times \{0, \ldots, l-1\}$.
The interesting fact is that we can distribute the sampling operation over composition, e.g.:
\begin{isabelle_cm}
  sample\ \isacharparenleft\isasymlambda\isasymomega\isachardot\ f\ \isasymomega\ \isasymcirc\ g\ \isasymomega{\isacharparenright}\ \isacharequal\ sample\ g\ \isasymbind\ \isacharparenleft{\isasymlambda}x\isachardot\ sample\ \isacharparenleft\isasymlambda\isasymomega\isachardot\ f\ \isasymomega\isachardot\ x\isacharparenright\isacharparenright
\end{isabelle_cm}
if $f$ and $g$ depend on disjoint subsets of the coin flips.




\todo{Explain how we formalized the original proof by Chakraborty et al. + brief section on the lazify}  



\section{Related Work}\label{sec:related_work}
\subsection{Algorithms for the Distinct Elements Problem}
It is important to note that there are several practical solutions for the distinct elements problem.
The first solution was presented by Flajolet~\cite{flajolet1985} in 1985; however, like many other authors~\cite{flajolet2007,heule2013,pettie2021}, his solution makes the assumption that a fixed hash function can be regarded as a fully random function.
Alon et al.~\cite[Section 2.3]{alon1999} presented an easy remedy, which does not require such unmotivated model assumptions.
Their algorithm just relies on keeping track of the maximum of the hash values of the stream elements, where the hash function must be chosen uniformly from a pairwise independent family; the space complexity of this algorithm is $\bigo(\varepsilon^{-2} \ln (\delta^{-1}) b)$, where we assume that $b$ is the number of bits required to represent the stream elements.
This is slightly better than the CVM algorithm which requires $\bigo(\varepsilon^{-2} \ln (\delta^{-1}l) b)$ since there is no logarithmic dependency on $l$.

Later, Bar-Yossef et al.~\cite{baryossef2002}, Kane et al.~\cite{kane2010} and B\l{}asiok in 2020~\cite{blasiok2020} presented increasingly sophisticated solutions.
The last one by B\l{}asiok is optimal, with a space complexity of $\bigo(\varepsilon^{-2} \ln (\delta^{-1}) + b)$.
Karayel~\cite{karayel2023} presented a version of the latter that preserves monotonicity and supports the merge-operation, which enables its use in distributed settings, such as Map-Reduce pipelines~\cite{dean2010}.
It should be noted that these recent algorithms are mostly of theoretical interest, as the constants, as well as the implementation complexity is rather large.
What makes the CVM algorithm unique is its simplicity and the fact that it does not rely on hashing, which may enable more general use-cases than the traditional algorithms.

The aforementioned hash-based algorithms are biased; Flajolet et al.~\cite{flajolet1985} points this out and also provides bounds on the distance between the expected result and the cardinality of the stream.
Most authors do not discuss the matter of biasedness but it is not hard to show.
One issue, for example, is that the usual method to amplify the accuracy of these algorithms is using the median, which does not preserve expectations.
In the context of query processing, unbiasedness has been discussed~\cite[Section 2.1]{haas1995}, but we could not find any similar discussion for the distinct elements problem in the streaming model.

\subsection{Probabilistic Invariants and Formalization}
As far as we know, probabilistic invariants have not been used to establish Gaussian tail-bounds.
However, it is fairly common to establish results about expectations or variance of random variables, such as their run-time~\cite[Section 1.4]{motwani1995}, using recursive analysis techniques.
This is easy due to the linearity of expectations and---for independent random variables---variances.
A simple example is the Morris-counter~\cite{morris1978} or the expected run-time of the quick-sort algorithm~\cite[Section 2.5]{mitzenmacher2005}.

There is also research on the (automated) analysis of loop invariants, for probabilistic loops, using their characteristic functions~\cite{batz2023, mciver2005}.
This approach works by establishing the limiting distribution of the state of the loop.
De Medeiros et al.~\cite[Section 3.2]{demedeiros2024} also establish methods to derive limiting distributions of probabilistic loops.
Our approach differs from these techniques, by avoiding computation of the distribution, which, we think, is infeasible for the CVM algorithm.
Instead, we investigate classes of functions of the distributions, which are relevant for the analysis.
There is research on automated evaluation of moments for restricted classes of loops, which contain only polynomial assignments and no branches~\cite{bartocci2019,kofnov2022}.
However, these methods, do not extend to algorithms with branches or, more generally, algorithms which contain discrete operations.

Finally, verification of randomized algorithms has been tackled by authors using various proof assistants~\cite{bosshard2024,demedeiros2024, eberl2020,gopinathan20,hurd03, Probabilistic_Prime_Tests-AFP, tan2024}.
The most closely related efforts are the mechanizations of frequency moments algorithms by Karayel~\cite{karayel2022, karayel2023}.
The functional invariant proof technique we introduce here should be applicable in any higher-order setting.


\section{Conclusion}\label{sec:conclusion}
We presented the first formalization of the CVM algorithm using Isabelle/HOL.
Central to our formalization is a novel invariant-based proof technique for randomized algorithms which is inspired by our alternative analysis of the CVM algorithm via the Cram\'{e}r--Chernoff method.
Interestingly, the technique readily generalized to a new CVM variant with stronger properties (totality and unbiasedness)---we formalized this latter version using the same invariant, together with a new library of results for negative association.
In future work, it would be interesting to formalize other variations of subsampling for CVM.
More generally, one could further explore whether the technique we introduced here could be applied towards proofs of other randomized (streaming) algorithms.


%%
%% Bibliography
%%

%% Please use bibtex,

\bibliography{main}

\end{document}
