
\documentclass[a4paper,UKenglish,cleveref, autoref, thm-restate]{lipics-v2021}
%This is a template for producing LIPIcs articles.
%See lipics-v2021-authors-guidelines.pdf for further information.
%for A4 paper format use option "a4paper", for US-letter use option "letterpaper"
%for british hyphenation rules use option "UKenglish", for american hyphenation rules use option "USenglish"
%for section-numbered lemmas etc., use "numberwithinsect"
%for enabling cleveref support, use "cleveref"
%for enabling autoref support, use "autoref"
%for anonymousing the authors (e.g. for double-blind review), add "anonymous"
%for enabling thm-restate support, use "thm-restate"
%for enabling a two-column layout for the author/affilation part (only applicable for > 6 authors), use "authorcolumns"
%for producing a PDF according the PDF/A standard, add "pdfa"

%\pdfoutput=1 %uncomment to ensure pdflatex processing (mandatatory e.g. to submit to arXiv)
%\hideLIPIcs  %uncomment to remove references to LIPIcs series (logo, DOI, ...), e.g. when preparing a pre-final version to be uploaded to arXiv or another public repository

\usepackage{algorithm}
\usepackage[noend]{algpseudocode}
\renewcommand{\algorithmicrequire}{\textbf{Input:}}
\renewcommand{\algorithmicensure}{\textbf{Output:}}

\newcommand{\getsr}{\xleftarrow{\$}}

\usepackage{mathtools}
\DeclarePairedDelimiter{\ceil}{\lceil}{\rceil}
\DeclareMathOperator{\Ber}{Ber}

%\graphicspath{{./graphics/}}%helpful if your graphic files are in another directory

\bibliographystyle{plainurl}% the mandatory bibstyle

\title{Formalization of CVM} %TODO Please add

%\titlerunning{Dummy short title} %TODO optional, please use if title is longer than one line

\author{Jane {Open Access}}{Dummy University Computing Laboratory, [optional: Address], Country \and My second affiliation, Country \and \url{http://www.myhomepage.edu} }{johnqpublic@dummyuni.org}{https://orcid.org/0000-0002-1825-0097}{(Optional) author-specific funding acknowledgements}%TODO mandatory, please use full name; only 1 author per \author macro; first two parameters are mandatory, other parameters can be empty. Please provide at least the name of the affiliation and the country. The full address is optional. Use additional curly braces to indicate the correct name splitting when the last name consists of multiple name parts.

\author{Joan R. Public\footnote{Optional footnote, e.g. to mark corresponding author}}{Department of Informatics, Dummy College, [optional: Address], Country}{joanrpublic@dummycollege.org}{[orcid]}{[funding]}

\authorrunning{J. Open Access and J.\,R. Public} %TODO mandatory. First: Use abbreviated first/middle names. Second (only in severe cases): Use first author plus 'et al.'

\Copyright{Jane Open Access and Joan R. Public} %TODO mandatory, please use full first names. LIPIcs license is "CC-BY";  http://creativecommons.org/licenses/by/3.0/

\ccsdesc[100]{\textcolor{red}{Replace ccsdesc macro with valid one}} %TODO mandatory: Please choose ACM 2012 classifications from https://dl.acm.org/ccs/ccs_flat.cfm

\keywords{Dummy keyword} %TODO mandatory; please add comma-separated list of keywords

\category{} %optional, e.g. invited paper

\relatedversion{} %optional, e.g. full version hosted on arXiv, HAL, or other respository/website
%\relatedversiondetails[linktext={opt. text shown instead of the URL}, cite=DBLP:books/mk/GrayR93]{Classification (e.g. Full Version, Extended Version, Previous Version}{URL to related version} %linktext and cite are optional

%\supplement{}%optional, e.g. related research data, source code, ... hosted on a repository like zenodo, figshare, GitHub, ...
%\supplementdetails[linktext={opt. text shown instead of the URL}, cite=DBLP:books/mk/GrayR93, subcategory={Description, Subcategory}, swhid={Software Heritage Identifier}]{General Classification (e.g. Software, Dataset, Model, ...)}{URL to related version} %linktext, cite, and subcategory are optional

%\funding{(Optional) general funding statement \dots}%optional, to capture a funding statement, which applies to all authors. Please enter author specific funding statements as fifth argument of the \author macro.

\acknowledgements{I want to thank \dots}%optional

%\nolinenumbers %uncomment to disable line numbering

%Editor-only macros:: begin (do not touch as author)%%%%%%%%%%%%%%%%%%%%%%%%%%%%%%%%%%
\EventEditors{John Q. Open and Joan R. Access}
\EventNoEds{2}
\EventLongTitle{42nd Conference on Very Important Topics (CVIT 2016)}
\EventShortTitle{CVIT 2016}
\EventAcronym{CVIT}
\EventYear{2016}
\EventDate{December 24--27, 2016}
\EventLocation{Little Whinging, United Kingdom}
\EventLogo{}
\SeriesVolume{42}
\ArticleNo{23}
%%%%%%%%%%%%%%%%%%%%%%%%%%%%%%%%%%%%%%%%%%%%%%%%%%%%%%

\begin{document}

\maketitle

%NOTE: we have to inline all of these into one file for the final version if accepted.

\begin{abstract}
Estimating the number of distinct elements in a data stream is a classical problem with numerous applications in computer science.
We formalize a recent, remarkably simple, randomized algorithm for this problem due to Chakraborty, Vinodchandran, and Meel (called the CVM algorithm).
Their algorithm deviated considerably from the state of the art, due to its avoidance of intricate derandomization techniques, while still maintaining a close-to-optimal logarithmic space complexity.

Central to our formalization is a new proof technique based on \emph{functional probabilistic invariants}, which allows us to derive concentration bounds using the Cram\'{e}r--Chernoff method.
This simplifies the formal analysis considerably compared to the original proof by Chakraborty et al.
Moreover, our technique opens up the possible algorithm design space; we demonstrate this by introducing and verifying a new variant of the CVM algorithm that is total and unbiased---neither property is possessed by the original algorithm.
In this paper, we introduce the proof technique, describe its use in mechanizing both versions of the CVM algorithm in Isabelle/HOL, and present a supporting formalized library on negatively associated random variables used to verify the latter variant.
\end{abstract}




\section{Introduction}
\label{sec:intro}

In 2022, Chakraborty, Vinodchandran, and Meel~\cite{chakraborty2022} published a marvelous streaming algorithm for the distinct elements problem, which was very unexpected in the community~\cite{quanta}.
Indeed, Knuth later wrote a note on the algorithm~\cite{knuthnote}, pointing out its interesting properties and christening it the \emph{CVM} algorithm (which we use for the rest of this paper).
One striking property of the CVM algorithm is that, in contrast to every other known algorithm for the problem, it does not rely on hashing the stream elements.
Instead, the algorithm could theoretically be implemented in a setting where objects in the data stream only allow for equality comparisons.
Another property is its simplicity---which is why the authors called it ``an algorithm for the text book''---the algorithm is shown in its entirety in~\cref{alg:cvm}.

\begin{algorithm}[h!]
	\caption{CVM algorithm for distinct elements estimation~\cite{chakraborty2022}.}\label{alg:cvm}
	\begin{algorithmic}[1]
  \Require Stream elements $a_1,\dots,a_l$, $0 < \varepsilon$, $0 < \delta < 1$.
  \Ensure A cardinality estimate $R$ for set $A = \{ a_1,\dots,a_l \}$ s.t. $\prob \left( |R - |A| | > \varepsilon |A| \right) \leq \delta$
  \State $\chi \gets \{\}, p \gets 1, n = \ceil*{\frac{12}{\varepsilon^2} \ln{(\frac{6l}{\delta})} }$
  \For{$i \gets 1$ to $l$}
    \State $b \getsr \Ber(p)$ \Comment insert $a_i$ with probability $p$ (and remove it otherwise)
    \If{$b$}
      \State $\chi \gets \chi \cup \{a_i\}$
    \Else
      \State $\chi \gets \chi - \{a_i\}$
    \EndIf
    \If{$|\chi| = n$}
      \State $\chi \getsr \mathrm{subsample}(\chi)$ \Comment discard elements of $\chi$ independently with prob. $\frac{1}{2}$
      \State $p \gets \frac{p}{2}$
    \EndIf
    \If{$|\chi| = n$}
      \Return $\bot$ \Comment fail if $\chi$ remains too large
    \EndIf
  \EndFor
  \State \Return $\frac{|\chi|}{p}$ \Comment estimate cardinality of $A$
  \end{algorithmic}
\end{algorithm}

The pen-and-paper analysis of CVM~\cite{chakraborty2022,chakraborty2023} relies on a sequence of transformations of the algorithm.
The reason for these transformations is that standard methods for analyzing randomized algorithms, such as Chernoff--Hoeffding bounds, usually make statements about independent random variables.
Yet, for \cref{alg:cvm}, the state variables are far from being independent.\footnote{There is an incorrect claim in the initial published proof of CVM~\cite[Claim 6]{chakraborty2022} that the indicator functions for elements in $\chi$ are independent; a later version by the same authors~\cite{chakraborty2023} provides a correct proof.
The original error serves as a side motivation for this work.}
For example, in line 3 the Bernoulli distribution is sampled for a value $p$, which itself depends on previous random operations; similarly, the subsampling step in line 9 is only applied if the buffer is full, which also depends on previous random operations.
The aforementioned sequence of transformations results in another randomized algorithm which can be analyzed using standard methods, from which the estimation results for the original algorithm can be deduced.
To our knowledge, it seems impossible to analyze \cref{alg:cvm} more directly using known methods.

In this paper, we present a new technique for analyzing randomized algorithms which yields a direct and substantially more general proof of the CVM algorithm.
Our approach is very similar to how deterministic algorithms are verified using a loop invariant.
The key difference is that our choice of ``loop invariant'' for the randomized streaming algorithm is a functional probabilistic inequality, namely, we consider invariants of the form:
\[
  \expect [ h ] \leq h(c)
\]
where $h$ is allowed to range over a class of functions, and the expectation is taken over the distribution of the state of the algorithm after consuming each stream element.
By first establishing such an invariant for \cref{alg:cvm}, we can then use it (via different choices of $h$) to establish error bounds for the algorithm.
For the rest of this paper, we explain this technique, its mechanization, and show how it leads to a proof of the CVM algorithm.
We believe the new proof remains accessible at the undergraduate level, albeit with some exposure to ideas from interactive theorem proving.

The main contributions of this work are:
\begin{itemize}
\item Introduction of a new technique using functional probabilistic invariants to verify randomized algorithms inductively/recursively.
\item Verification of the original CVM algorithm using our new technique.
\item Presentation and verification of a new variant of CVM that is total and unbiased.
\item Formalization of a theory of negatively associated random variables used to analyze the new CVM variant.
\end{itemize}

The verification of the CVM algorithm using our new technique requires only 1044 lines in Isabelle~\cite{nipkow2002}, while the original proof (which we also mechanized) required \todo{x} lines.
We detail some of the challenges faced when mechanizing the transformation arguments used by Chakraborty et al.~\cite{chakraborty2022,chakraborty2023} in \cref{sec:transformation_based_proof}.

To show that our new technique is more general, we introduce a new variant of the CVM algorithm, where the subsampling step in line 9 of \cref{alg:cvm} selects a random $m$-subset of $\chi$, instead of independently deciding for each element, whether to keep it.
This variant has the benefit that it is \emph{total} (never returns $\bot$) because the second check in line 11 becomes obsolete.
More interestingly, the resulting variant is \emph{unbiased}, i.e., the expected result is exactly the cardinality of the elements in the stream; this is a new property, that neither the original CVM algorithm nor classic algorithms for the distinct elements problem possess.

To analyze the new variant, we use results from the theory of negatively dependent random variables~\cite{joagdev1983} to establish the desired functional invariant.
The concept of negative association is a generalization of independence; negatively associated variables observe closure properties and fulfill Chernoff--Hoeffding bounds similar to independent random variables.
It should be stressed that the theory of negatively associated RVs is orthogonal to our new technique, but the formalization of such a theory is also a contribution of this work.

\cref{sec:background} provides background information on randomized algorithms, in particular on their semantics in Isabelle~\cite{nipkow2002}.
\cref{sec:invariants} introduces our new technique, and shows how probabilistic loop invariants can be used to establish tail bounds for the original CVM algorithm.
\cref{sec:negdep} introduces the concept of negative association and our new total and unbiased variant of the CVM algorithm.
\cref{sec:formalization} and \cref{sec:formalization_neg_dep} present details about the formal verification of both variants of the algorithm, as well as, our new library on negatively associated random variables.
In \cref{sec:transformation_based_proof}, we present details about our alternative verification of the CVM using the transformation based proof by Chakraborty et al.
The last sections conclude with a brief overview of related work and a summary of our results.

% TODO: move it into the discussion of the variant, I think it is better once the algorithm is shown
%This is because indicator functions of $n$-subsets form negatively associated random variables (RV), even though they are not independent, with which we can modify the subsampling step to the form we described above.

%TODO: I think this should be later, we can give a forward reference
%% The algorithm's state is a buffer $\chi$  (initially empty) and a fraction $p > 0$ (initially set to $1$).
%% The buffer contains a subset of the elements of the stream encountered so far, with maximal size $n$.
%% The size is chosen according to the desired accuracy parameters $\varepsilon$, $\delta$, and the stream size $l$.
%% The algorithm iterates over stream elements, adding each one to the buffer with probability $p$ or conversely -- if the current stream element is already in the buffer -- removing them with probability $(1-p)$.
%% If the buffer gets too large, approximately half of the elements are removed by discarding each element in $\chi$ independently with probability $\frac{1}{2}$; then, $p$ is adjusted to reflect the fact that the buffer now contains each element with probability $p_\text{new} = \frac{p_\text{old}}{2}$.
%% After processing the stream, the algorithm returns $\frac{|\chi|}{p}$ as an approximation of the number of distinct elements in the stream.

%% %TODO: a bit too long in details, we should trim this and move to later section
%% \subparagraph*{The road not taken:}
%% One of the difficult aspects of the proof is that it relies on an eager-lazy coin flip conversion.
%% To understand that, we should note that none of the observable random variables, such as the presence of a stream element in the buffer or conditions on the value of $p$, are independent of the other state variables, which makes the algorithm hard to analyze and makes the application of standard techniques from probability theory, such as Hoeffding's theorem impossible.
%% The authors resolved that problem by a simulation argument---they show that~\cref{alg:cvm} behaves stochastically identically to a different algorithm, which makes the relevant coin flips in a different order.
%% That modified algorithm performs a column of coin flips for each stream element.
%% An element is kept in the buffer if the first $k=\log_2(p-1)$ rows of the column are heads.
%% At each sub-sampling step, when p is divided by two, i.e., if k increases by one.
%% The algorithm examines the newly activated $k$-th row of the previous sequence elements to decide whether the element should be kept in the buffer.
%% This preserves the invariant that the buffer consists of exactly those sequence elements whose associated coin flip column starts with $\log_2(p-1)$ heads.
%% Of course, the new algorithm is not practical for actual implementation, but one can verify its correctness using standard Chernoff bounds, and on the other hand, it is possible to show that its behavior is equivalent to Algorithm 1.
%% To summarize, while the algorithm is marvelous, the proof was still very technical.
%% The simulation argument, in particular, is not so elegant to formalize.

%% \subparagraph*{A more direct proof:}
%% We set out to try to find a more direct proof, which also eases the formalization effort.
%% For the following discussion, we will analyze Algorithm 1 with line 8 removed, i.e., the algorithm does not output $\bot$, nor performs a second check of $|\chi|=n$.
%% Note that this happens if the very improbable event where none of the elements in X are removed during a subsampling step, which happens with probability at most $2^{-n/2}$.
%% Overall, the probability of it happening during the course of the algorithm is at most $\frac{\delta}{2}$. (Note that removing line 8 does not affect the correctness of the algorithm, but it loses its space consumption bound.)
%% It is easy to see that any probability established about the algorithm missing line 8 will be true for the original algorithm with a possible correction by, at most,  $\frac{\delta}{2}$.
%% We will remember and correct this at the end of this section.

%% Let us consider an imaginary situation where, somehow, $p$ is fixed at some point in the algorithm.
%% For example, we could imagine a final sub-sampling loop, which is run until a fixed $p$ is reached.
%% Then, the indicator random variables representing the presence of a stream element TODO


\section{An Unbiased CVM Variant and Negative Dependence\label{sec:negdep}}

An interesting consequence of our invariant-based approach is that it allowed us to devise and verify a refined version of the CVM algorithm that is both total and unbiased.

\subsection{Unbiased CVM Variant}

\begin{algorithm}[t!]
	\caption{New total and unbiased CVM algorithm variant.}\label{alg:cvm_new}
	\begin{algorithmic}[1]
  \Require Stream elements $a_1,\dots,a_l$, $0 < \varepsilon$, $0 < \delta < 1$.
  \Ensure A cardinality estimate $R$ for set $A = \{ a_1,\dots,a_l \}$ s.t. $\prob \left( |R - |A| | > \varepsilon |A| \right) \leq \delta$
  \State $\chi \gets \{\}, p \gets 1, n = \ceil*{\frac{12}{\varepsilon^2} \ln{(\frac{3l}{\delta})} }$, $\frac{1}{2} \leq f < 1$, s.t., $nf$ integer
  \For{$i \gets 1$ to $l$}
    \State $b \getsr \Ber(p)$ \Comment insert $a_i$ with probability $p$ (and remove it otherwise)
    \If{$b$}
      \State $\chi \gets \chi \cup \{a_i\}$
    \Else
      \State $\chi \gets \chi - \{a_i\}$
    \EndIf
    \If{$|\chi| = n$} \Comment if buffer $\chi$ is full
      \State $\chi \getsr \mathrm{subsample}(\chi)$ \Comment select a random $nf$-subset of $\chi$
      \State $p \gets pf$
    \EndIf
  \EndFor
  \State \Return $\frac{|\chi|}{p}$ \Comment estimate cardinality of $A$
  \end{algorithmic}
\end{algorithm}%


When we look at the subsampling step of~\cref{alg:cvm}, our invariant~\cref{i:func_invariant} imposes the following condition on the subsampling operation:
\begin{equation}\label[ineq]{i:subsample_condition}
  \int_{\mathrm{subsample}(\chi)} \prod_{s \in S} g(\indicat(s \in \tau)) \, d \tau \leq \prod_{s \in S} \expect_{\Ber(f)} [g]
\end{equation}
for all non-negative functions $g$ and any $S \subseteq \chi$.
Any subsampling step that satisfies this functional inequality can be used while still preserving~\cref{i:func_invariant} for the algorithm.

Motivated by this observation, our new variant is shown in \cref{alg:cvm_new}.
For the subsampling step, instead of keeping each element of $\chi$ with probability $\frac{1}{2}$, we instead pick a uniform random $nf$-subset of $\chi$, where $\frac{1}{2} \leq f < 1$ and such that $nf$ is an integer.
For example, it is possible to choose $f = \frac{n-1}{n}$, i.e., discarding a random element from $\chi$ in the subsampling step.
Since this new subsampling step always reduces the size of $\chi$, the variant is \emph{total} (never returns $\bot$).
The invariant-based approach allows us to show that the algorithm is probably-approximately correct and also \emph{unbiased}, i.e., the expectation of the result is exactly $|A|$.
These depend crucially on establishing~\cref{i:subsample_condition} for the new subsampler, for which we need a new concept.

\subsection{Background on Negative Dependence}
Some sets of random variables possess a property called \emph{negative association}, a generalization of independence.
The concept was introduced by Joag-Dev and Proschan~\cite{joagdev1983}, who showed that it has many useful closure properties compared to other previously introduced notions of negative dependence, such as negative correlation or negative orthant dependence. %TODO: perhaps citations for these two?
Importantly, standard Chernoff--Hoeffding type bounds still apply to negatively associated random variables~\cite[Prop. 7]{dubhashi1998}.
Negative association is defined as follows:
\begin{definition}
For a function defined on $n$-tuples $f: V^n \rightarrow W$, we will denote by $\mathrm{dep}(f)$ the set of coordinates the function depends on, i.e., $dep(f) \subseteq \{1,\ldots,n\}$ is minimal, s.t., $f(x) = f(y)$ for all $x, y \in V^n$ with $x_i = y_i$ for all $i \in dep(f)$.
\end{definition}

\begin{definition}[Negative Association]\label{def:neg_assoc}
A set of random variables $X_1,\dots,X_n: \Omega \rightarrow \mathbb R$ is negatively associated if, for all non-decreasing functions $f,g: \mathbb R^n \rightarrow \mathbb R$, which depend on disjoint sets of the variables, i.e., $\mathrm{dep}(f) \cap \mathrm{dep}(g) = \emptyset$, the following inequality holds:
\[
\expect [f(X_1,\ldots,X_n) g(X_1,\ldots,X_n)] \leq \expect [f(X_1,\ldots,X_n)] \expect [g(X_1,\ldots,X_n)] \textrm{.}
\]
\end{definition}

The following proposition summarizes some important properties of negatively associated sets of random variables.
\begin{proposition}[\cite{joagdev1983}]\label{pro:neg_dep_props}
Summary of results for negatively associated random variables.
\begin{enumerate}
\item \label{it:neg_dep_props:mult_mono} If $X=(X_1,\ldots,X_n)$ are negatively associated then $\expect [f(X) g(X)] \leq \expect [f(X)] \expect [g(X)]$ for non-increasing functions $f,g$ with $\mathrm{dep}(f) \cap \mathrm{dep}(g) = \emptyset$.
\item If $X=(X_1,\ldots,X_n)$ are negatively associated, $Y=(Y_1,\ldots,Y_m)$ are negatively associated, and the pair of vector-valued random variables $X$ and $Y$ are independent, then the union $X_1,\dots,X_n,Y_1,\dots,Y_m$ is a set of negatively associated random variables.
\item If $X=(X_1,\ldots,X_n)$ are negatively associated and $f_1, \dots ,f_m : \mathbb R^n \rightarrow \mathbb R$ are all non-increasing or all non-decreasing functions, s.t., $\mathrm{dep}(f_i) \cap \mathrm{dep}(f_j) = \emptyset$ for $i \neq j$, then $f_1(X),\ldots,f_m(X)$ form a set of negatively dependent random variables of size $m$.
\item If $X_1,\dots,X_n$ are independent then $X_1,\dots,X_n$ are negatively associated.
\item A subset of a negatively associated set of random variables is again negatively associated.
\end{enumerate}
\end{proposition}

These properties illustrate the trade-off between negative association and independence.
For example, Property 3 would be true for independent random variables, even without the condition of monotonicity.
%Of course, on the other hand independence is a stronger property and fewer sets of random variables are independent.
To analyze our new subsampler, the following is an important lemma about negative associated random variables.

\begin{lemma}\label{le:neg_assoc_prod}
Let $X_1,\dots,X_n$ be negatively associated and $f_1,\dots,f_n$ be all non-decreasing (or all non-increasing), non-negative functions, then
\[
  \expect \left[\prod_{i=1}^{n} f_i(X_i)\right] \leq \prod_{i=1}^{n} f_i(\expect [X_i]) \textrm{.}
\]
\end{lemma}
\begin{proof}
This follows from the definition of negative associativity (or Property~1 of \cref{pro:neg_dep_props}, if the $f_i$ are non-increasing) using induction.
\end{proof}

The case for non-decreasing functions of the above lemma is pointed out by Joag-Dev and Proschan~\cite[P.2]{joagdev1983}.
The reason for our interest in this lemma stems from the fact that indicator variables of random $m$-subsets are negatively associated.
This is a consequence of the fact that permutation distributions are negatively-associated~\cite[Th. 2.11]{joagdev1983}.
Thus, for the new subsampling step in Line 9 of~\cref{alg:cvm_new}, we can derive using \cref{le:neg_assoc_prod}:
\begin{equation}\label{eq:subsample_with_n_subsets}
  \int_{\mathrm{subsample}(\chi)} \prod_{s \in S} g(\indicat(s \in \tau))\,d\tau \leq
  \prod_{s \in S} \int_{\mathrm{subsample}(\chi)} g(\indicat(s \in \tau))\,d\tau = \prod_{s \in S}  \expect_{\Ber(f)}[g] \textrm{.}
\end{equation}
for any non-negative $g$ and $S \subseteq \chi$.
Note that the domain of $g$ has two values, so it is either non-increasing or non-decreasing.
Also, if $S$ is a singleton, the inequality becomes an equality.
With this ingredient, we can conclude that our results about the original algorithm derived in the previous section also hold for our new variant (\cref{alg:cvm_new}).
% Note: After reading this section. It stops too abrubtly at the previous sentence, without establishing what we have actually shown.


\section{A Simpler CVM Algorithm}
\label{sec:cvmalt}

TODO


%%
%% Bibliography
%%

%% Please use bibtex,

\bibliography{main}

\end{document}
