\documentclass[11pt, a4paper]{report}

\usepackage[T1]{fontenc}
\usepackage{isabelle, isabellesym}
\usepackage{amsfonts, amsmath, amssymb}
\usepackage[htt]{hyphenat}

% this should be the last package used
\usepackage{pdfsetup}

% urls in roman style, theory text in math-similar italics
\urlstyle{rm}
\isabellestyle{it}

\begin{document}

\title{Verification of the CVM algorithm}

\author{
  Seng Joe Watt,
  Derek Khu,
  Emin Karayel,
  Kuldeep S. Meel,
  \\
  and Yong Kiam Tan
}

\maketitle

\begin{abstract}
  Published in 2022, the CVM algorithm \cite{cvm_2022} is a remarkably simple
  one for the distinct elements problem, avoiding intricate derandomization
  techniques, while maintaining a close to optimal logarithmic space complexity.
  Later, it was discovered that the proof was incorrect, with the authors
  presenting a new proof in \cite{cvm_2023}.

  This project formalises the new, corrected proof of the CVM algorithm, while
  strengthening the correctness result of Theorem 2 in \cite{cvm_2023} to provide
  the same correctness guarantee using a smaller threshold
  (and hence space usage) of
  $\lceil \frac{12}{\varepsilon^2} \text{log}_2 \frac{3m}{\delta} \rceil$,
  where $m$ is the length of the input stream.

  This corrected proof performs a sequence of transformations from the CVM
  algorithm, akin to cryptographic game-hopping.
  The reason for these transformations is that standard techniques for analysing
  randomised algorithms such as the Chernoff bounds usually require independence
  of state random variables, which is unfortunately absent from the CVM
  algorithm, but was incorrecty claimed and used in the original proof.
  The transformations address this by successively simplifying the
  algorithm into one where the state variables are independent, enabling
  the use of the Chernoff bounds.

  As part of the project, we also formalise various useful results, in particular:
  \begin{enumerate}
    \item Unary and relational Hoare rules for reasoning about monadic folds
    (ie. loops) over the \texttt{reader}, \texttt{pmf} and \texttt{spmf} monads,
    which we use to justify the various program transformations.

    \item Strong forms of the multiplicative Chernoff bounds for the binomial
    distribution, derived from the Bennet-Bernstein inequalities.
  \end{enumerate}
\end{abstract}

\tableofcontents

\parindent 0pt
\parskip 0.5ex

\chapter{CVM algorithm formalisation}

\input{CVM_Algo}
\input{CVM_Algo_No_Fail}
\input{CVM_Algo_Lazy_Eager}
\input{CVM_Algo_Nondet_Binomial}

\input{CVM_Error_Bounds}
\input{CVM_Correctness}
\input{CVM_Correctness_Instance}

\chapter{Utility definitions and results}

\input{Utils_Basic}
\input{Utils_Reader_Monad}
\input{Utils_Approx_Algo}
\input{Utils_Concentration_Ineqs}

\input{Utils_PMF_Basic}
\input{Utils_PMF_Lazify}

\input{Utils_SPMF_Basic}
\input{Utils_SPMF_Rel_Hoare}

\bibliographystyle{abbrv}
\bibliography{root}

\end{document}