\documentclass[11pt,a4paper]{article}
\usepackage{isabelle,isabellesym}
\usepackage{amsfonts,amsmath, amssymb}
\usepackage[top=1.4in, bottom=1.4in, left=1.4in, right=1.4in]{geometry}

\usepackage[only,bigsqcap]{stmaryrd}

\usepackage{algorithm}
\usepackage[noend]{algpseudocode}
\renewcommand{\algorithmicrequire}{\textbf{Input:}}
\renewcommand{\algorithmicensure}{\textbf{Output:}}
\let\oldsection\section
\renewcommand\section{\clearpage\oldsection}

\newcommand{\getsr}{\xleftarrow{\$}}

\usepackage{mathtools}
\DeclarePairedDelimiter{\ceil}{\lceil}{\rceil}
\DeclareMathOperator{\prob}{\mathcal P}
\DeclareMathOperator{\expect}{\mathbb E}
\DeclareMathOperator{\Ber}{\mathrm{Ber}}
\DeclareMathOperator{\indicator}{\mathrm{I}}

% this should be the last package used
\usepackage{pdfsetup}

% urls in roman style, theory text in math-similar italics
\urlstyle{rm}
\isabellestyle{it}

\begin{document}

\title{Verification of the CVM algorithm with a new Recursive Analysis Technique}
\author{Emin Karayel, Derek Khu, Kuldeep S. Meel, Yong Kiam Tan,\\and Joe Watt}
\maketitle

\begin{abstract}
In 2022, Chakraborty et al.~\cite{chakraborty2022} published a streaming algorithm for the distinct
elements problem, that deviated considerably from the state-of-the art, both due to its simplicity
and avoidance of standard derandomization techniques, while still maintaining a close to optimal
logarithmic space complexity.

In this entry, we verify its correctness using a new verification technique we discovered during the
effort, that simplifies the analysis considerably compared to the orignal proof by Chakraborty et
al. The main idea is based on the well-known notion of probabilistic recurrence relations. However,
instead of recurrence relations with respect to the state, we investigate general functions of the
state and derive conditions on the function, which preserve the reccurence relation.

The latter allows us to derive concentration bounds which is usually not possible with 
recurrence relations.

This new technique also opened up the possible design space, and we introduce a new variant of the 
CVM algorithm, that uses less space, fewer steps, and also has an additional property in addition 
to concentration: unbiasedness. This means the expected result of the algorithm is exactly equal to
the desired result. The latter is also a new property, that classic algorithms for the distinct
elements problem do not posses.
\end{abstract}

\tableofcontents

\parindent 0pt\parskip 0.5ex

\input{session}

\bibliographystyle{abbrv}
\bibliography{root}

\end{document}

